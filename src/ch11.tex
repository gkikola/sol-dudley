\chapter{Quadratic Congruences}

\section{Exercises}

\Exercise1
\label{exercise:quad-cong:2-3-1-cong-0}
Convert $2x^2 + 3x + 1 \equiv 0\pmod5$ to a quadratic congruence whose
first coefficient is $1$.
\begin{solution}
  Since $3\cdot2 = 6\equiv1\pmod5$, we may multiply the above
  congruence by $3$ to get
  \begin{equation*}
    x^2 + 4x + 3\equiv0\pmod5. \qedhere
  \end{equation*}
\end{solution}

\Exercise2
\label{exercise:quad-cong:2-3-1-cong-0-complete-sq}
Change the quadratic in Exercise~\ref{exercise:quad-cong:2-3-1-cong-0}
to the form (3).
\begin{solution}
  Completing the square gives
  \begin{equation*}
    x^2 + 4x + 4 \equiv 1\pmod5,
  \end{equation*}
  or
  \begin{equation*}
    (x + 2)^2 \equiv 1\pmod5. \qedhere
  \end{equation*}
\end{solution}

\Exercise3 By inspection, find all the solutions of the congruence in
Exercise~\ref{exercise:quad-cong:2-3-1-cong-0-complete-sq}.
\begin{solution}
  The congruence $x^2\equiv1\pmod5$ has solutions $x\equiv1$ and
  $x\equiv4$ (mod $5$), so
  \begin{equation*}
    (x + 2)^2\equiv1\pmod5
  \end{equation*}
  has solutions $x\equiv2$ and $x\equiv4\pmod5$.
\end{solution}

\Exercise4 If $p > 3$, what are the two solutions of
$x^2\equiv4\pmod{p}$?
\begin{solution}
  We have $p\mid(x-2)(x+2)$ so $p\mid(x-2)$ or $p\mid(x+2)$. Then
  $x\equiv2$ or $x\equiv p-2$ (mod $p$). By Theorem~1, these are the
  only solutions.
\end{solution}

\Exercise5 For what values of $a$ does $x^2\equiv a\pmod7$ have two
solutions?
\begin{solution}
  We find the values of $x^2$, reduced modulo $7$:
  \begin{center}
    \begin{tabular}{r|cccccc}
      $x$ & 1 & 2 & 3 & 4 & 5 & 6 \\
      $x^2\pmod7$ & 1 & 4 & 2 & 2 & 4 & 1
    \end{tabular}
  \end{center}
  From these values, we see that $x^2\equiv a\pmod7$ has two solutions
  when and only when $a = 1$, $2$, or $4$.
\end{solution}

\Exercise6 Find the solutions of $x^2\equiv8\pmod{31}$.
\begin{solution}
  One may check that this congruence satisfies Euler's Criterion. We
  have
  \begin{equation*}
    8\equiv39\equiv70\equiv101\equiv132
    \equiv2^2\cdot33\pmod{31},
  \end{equation*}
  and
  \begin{equation*}
    33\equiv64\equiv8^2\pmod{31}.
  \end{equation*}
  Therefore $8\equiv16^2\pmod{31}$ and we see that the quadratic
  congruence $x^2\equiv8\pmod{31}$ has the two solutions
  \begin{equation*}
    x\equiv-16\equiv15\pmod{31}
    \quad\text{and}\quad
    x\equiv16\pmod{31}. \qedhere
  \end{equation*}
\end{solution}

\Exercise7 What is $(1/3)$? $(1/7)$? $(1/11)$? In general, what is
$(1/p)$?
\begin{solution}
  Since $1$ is a quadratic residue mod $3$, the Legendre symbol
  $(1/3) = 1$. Similarly, $(1/7) = (1/11) = 1$. In general, for any
  odd prime $p$, $1$ satisfies Euler's Criterion so we have
  $(1/p) = 1$.
\end{solution}

\Exercise8 What is $(4/5)$? $(4/7)$? $(4/p)$ for any odd prime $p$?
\begin{solution}
  It is easy to see by Euler's Criterion that $(4/5) = (4/7) = 1$. And
  in fact, $2$ and $p - 2$ are always solutions to the quadratic
  congruence $x^2\equiv4\pmod{p}$ for any odd prime $p$. Hence
  $(4/p) = 1$.
\end{solution}

\Exercise9 Induce a theorem from the two preceding exercises.
\begin{solution}
  It seems that $(a^2/p) = 1$ for any $a$, provided $p\nmid
  a$. Indeed, this is easily seen to be true since $a$ itself is a
  solution to the congruence $x^2\equiv a^2\pmod{p}$.
\end{solution}
