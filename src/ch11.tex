\chapter{Quadratic Congruences}

\section{Exercises}

\Exercise1
\label{exercise:quad-cong:2-3-1-cong-0}
Convert $2x^2 + 3x + 1 \equiv 0\pmod5$ to a quadratic congruence whose
first coefficient is $1$.
\begin{solution}
  Since $3\cdot2 = 6\equiv1\pmod5$, we may multiply the above
  congruence by $3$ to get
  \begin{equation*}
    x^2 + 4x + 3\equiv0\pmod5. \qedhere
  \end{equation*}
\end{solution}

\Exercise2
\label{exercise:quad-cong:2-3-1-cong-0-complete-sq}
Change the quadratic in Exercise~\ref{exercise:quad-cong:2-3-1-cong-0}
to the form (3).
\begin{solution}
  Completing the square gives
  \begin{equation*}
    x^2 + 4x + 4 \equiv 1\pmod5,
  \end{equation*}
  or
  \begin{equation*}
    (x + 2)^2 \equiv 1\pmod5. \qedhere
  \end{equation*}
\end{solution}

\Exercise3 By inspection, find all the solutions of the congruence in
Exercise~\ref{exercise:quad-cong:2-3-1-cong-0-complete-sq}.
\begin{solution}
  The congruence $x^2\equiv1\pmod5$ has solutions $x\equiv1$ and
  $x\equiv4$ (mod $5$), so
  \begin{equation*}
    (x + 2)^2\equiv1\pmod5
  \end{equation*}
  has solutions $x\equiv2$ and $x\equiv4\pmod5$.
\end{solution}

\Exercise4 If $p > 3$, what are the two solutions of
$x^2\equiv4\pmod{p}$?
\begin{solution}
  We have $p\mid(x-2)(x+2)$ so $p\mid(x-2)$ or $p\mid(x+2)$. Then
  $x\equiv2$ or $x\equiv p-2$ (mod $p$). By Theorem~1, these are the
  only solutions.
\end{solution}

\Exercise5 For what values of $a$ does $x^2\equiv a\pmod7$ have two
solutions?
\begin{solution}
  We find the values of $x^2$, reduced modulo $7$:
  \begin{center}
    \begin{tabular}{r|cccccc}
      $x$ & 1 & 2 & 3 & 4 & 5 & 6 \\
      $x^2\pmod7$ & 1 & 4 & 2 & 2 & 4 & 1
    \end{tabular}
  \end{center}
  From these values, we see that $x^2\equiv a\pmod7$ has two solutions
  when and only when $a = 1$, $2$, or $4$.
\end{solution}

\Exercise6 Find the solutions of $x^2\equiv8\pmod{31}$.
\begin{solution}
  One may check that this congruence satisfies Euler's Criterion. We
  have
  \begin{equation*}
    8\equiv39\equiv70\equiv101\equiv132
    \equiv2^2\cdot33\pmod{31},
  \end{equation*}
  and
  \begin{equation*}
    33\equiv64\equiv8^2\pmod{31}.
  \end{equation*}
  Therefore $8\equiv16^2\pmod{31}$ and we see that the quadratic
  congruence $x^2\equiv8\pmod{31}$ has the two solutions
  \begin{equation*}
    x\equiv-16\equiv15\pmod{31}
    \quad\text{and}\quad
    x\equiv16\pmod{31}. \qedhere
  \end{equation*}
\end{solution}

\Exercise7 What is $(1/3)$? $(1/7)$? $(1/11)$? In general, what is
$(1/p)$?
\begin{solution}
  Since $1$ is a quadratic residue mod $3$, the Legendre symbol
  $(1/3) = 1$. Similarly, $(1/7) = (1/11) = 1$. In general, for any
  odd prime $p$, $1$ satisfies Euler's Criterion so we have
  $(1/p) = 1$.
\end{solution}

\Exercise8
\label{exercise:quad-cong:leg-sym-4-over-p}
What is $(4/5)$? $(4/7)$? $(4/p)$ for any odd prime $p$?
\begin{solution}
  It is easy to see by Euler's Criterion that $(4/5) = (4/7) = 1$. And
  in fact, $2$ and $p - 2$ are always solutions to the quadratic
  congruence $x^2\equiv4\pmod{p}$ for any odd prime $p$. Hence
  $(4/p) = 1$.
\end{solution}

\Exercise9 Induce a theorem from the two preceding exercises.
\begin{solution}
  It seems that $(a^2/p) = 1$ for any $a$, provided $p\nmid
  a$. Indeed, this is easily seen to be true since $a$ itself is a
  solution to the congruence $x^2\equiv a^2\pmod{p}$.
\end{solution}

\Exercise{10} Verify that
\begin{equation*}
  \text{if}\quad
  (a/p) = -1
  \quad\text{and}\quad
  a\equiv b\pmod{p},
  \quad\text{then}\quad
  (b/p) = -1.
\end{equation*}
\begin{proof}
  Suppose $(a/p) = -1$ and $a\equiv b\pmod{p}$, but that $(b/p) =
  1$. Then $x^2\equiv b\pmod{p}$ has a solution. But now
  $x^2\equiv a\pmod{p}$ must have the same solution, which gives a
  contradiction. Therefore $(b/p) = -1$.
\end{proof}

\Exercise{11} Prove that if $p\nmid a$, then $(a^2/p) = 1$, using the
fact that $(a/p)\equiv a^{(p-1)/2}\pmod{p}$.
\begin{proof}
  From the above, and by Fermat's Theorem, we have that
  \begin{equation*}
    (a^2/p) \equiv (a^2)^{(p-1)/2} \equiv a^{p-1} \equiv 1\pmod{p}.
  \end{equation*}
  Since the Legendre symbol on the left is either $1$ or $-1$, the
  congruence implies equality.
\end{proof}

\Exercise{12} Prove that $(4a/p) = (a/p)$.
\begin{proof}
  In Exercise~\ref{exercise:quad-cong:leg-sym-4-over-p} we saw that
  $(4/p) = 1$. So, by Theorem~3 (C), we have
  \begin{equation*}
    (4a/p) = (4/p)(a/p) = (a/p). \qedhere
  \end{equation*}
\end{proof}

\Exercise{13} Evaluate $(19/5)$ and $(-9/13)$ by using (A) and (B) of
Theorem~3.
\begin{solution}
  We have
  \begin{equation*}
    (19/5) = (4/5) = (2^2/5) = 1
  \end{equation*}
  and
  \begin{equation*}
    (-9/13) = (4/13) = (2^2/13) = 1. \qedhere
  \end{equation*}
\end{solution}

\Exercise{14} For which of the primes $3$, $5$, $7$, $11$, $13$, $17$,
$19$, and $23$ is $-1$ a quadratic residue?
\begin{solution}
  We can apply Theorem~5. Since $5\equiv13\equiv17\equiv1\pmod{4}$, we
  see that $-1$ is a quadratic residue (mod $p$) for $p = 5$, $13$,
  and $17$. It is not a quadratic residue for the remaining primes.
\end{solution}

\Exercise{15} Evaluate $(6/7)$ and $(2/23)(11/23)$.
\begin{solution}
  Note that $7\equiv3\pmod4$ and $23\equiv3\pmod4$. Therefore
  \begin{equation*}
    (6/7) = (-1/7) = -1
  \end{equation*}
  and
  \begin{equation*}
    (2/23)(11/23) = (22/23) = (-1/23) = -1. \qedhere
  \end{equation*}
\end{solution}

\section{Problems}

\Problem1
\label{problem:quad-cong:sample1}
Which of the following congruences have solutions?
\begin{align*}
  x^2 &\equiv 7\pmod{53} & x^2 &\equiv 14\pmod{31} \\
  x^2 &\equiv 53\pmod{7} & x^2 &\equiv 25\pmod{997}
\end{align*}
\begin{solution}
  By the Quadratic Reciprocity Theorem, we have
  \begin{equation*}
    (7/53) = (53/7) = (4/7) = 1.
  \end{equation*}
  Therefore $x^2\equiv7\pmod{53}$ and $x^2\equiv53\pmod7$ both have
  solutions.

  Next,
  \begin{equation*}
    (14/31) = (2/31)(7/31).
  \end{equation*}
  Since $31\equiv7\pmod8$, we have by Theorem~6 that $(2/31) = 1$. For
  $(7/31)$ we get, by Quadratic Reciprocity,
  \begin{equation*}
    (7/31) = -(31/7) = -(3/7) = (7/3) = (1/3) = 1.
  \end{equation*}
  Therefore $(14/31) = 1\cdot1 = 1$ and $x^2\equiv14\pmod{31}$ has a
  solution.

  Lastly, $(25/997) = (5^2/997) = 1$, so $x^2\equiv25\pmod{997}$ also
  has a solution.
\end{solution}

\Problem2
\label{problem:quad-cong:sample2}
Which of the following congruences have solutions?
\begin{align*}
  x^2 &\equiv 8 \pmod{53} & x^2 &\equiv 15\pmod{31} \\
  x^2 &\equiv 54 \pmod7 & x^2 &\equiv 625\pmod{9973}
\end{align*}
\begin{solution}
  $53\equiv5\pmod8$, so
  \begin{equation*}
    (8/53) = (2/53)(4/53) = (2/53) = -1,
  \end{equation*}
  and therefore the congruence $x^2\equiv8\pmod{53}$ has no solutions.

  Next,
  \begin{equation*}
    (15/31) = (3/31)(5/31) = -(31/3)(31/5)
    = -(1/3)(1/5) = -1,
  \end{equation*}
  so the congruence $x^2\equiv15\pmod{31}$ has no solutions.

  \begin{equation*}
    (54/7) = (5/7) = (7/5) = (2/5) = -1,
  \end{equation*}
  so $x^2\equiv54\pmod7$ has no solutions.

  Lastly,
  \begin{equation*}
    (625/9973) = (25^2/9973) = 1,
  \end{equation*}
  so $x^2\equiv625\pmod{9973}$ does have a solution.
\end{solution}

\Problem3 Find solutions for the congruences in
Problem~\ref{problem:quad-cong:sample1} that have them.
\begin{solution}
  For $x^2\equiv7\pmod{53}$, we have
  \begin{gather*}
    7\equiv60\equiv2^2\cdot15\pmod{53}, \\
    15\equiv68\equiv2^2\cdot17\pmod{53}, \\
    17\equiv70\equiv123\equiv176\equiv4^2\cdot11\pmod{53}, \\
    \intertext{and}
    11\equiv64\equiv8^2\pmod{53}.
  \end{gather*}
  Since $2\cdot2\cdot4\cdot8 = 128\equiv22\pmod{53}$, the congruence
  $x^2\equiv7\pmod{53}$ has the two solutions
  \begin{equation*}
    x\equiv22\pmod{53}
    \quad\text{and}\quad
    x\equiv31\pmod{53}.
  \end{equation*}

  Next, $53\equiv4\equiv2^2\pmod7$, so the congruence
  $x^2\equiv53\pmod7$ has the two solutions
  \begin{equation*}
    x\equiv2\pmod7
    \quad\text{and}\quad
    x\equiv5\pmod7.
  \end{equation*}

  Modulo $31$, we have
  \begin{equation*}
    14\equiv45\equiv3^2\cdot5\pmod{31}
  \end{equation*}
  and
  \begin{equation*}
    5\equiv36\equiv6^2\pmod{31}.
  \end{equation*}
  Since $3\cdot6 = 18$, the congruence $x^2\equiv14\pmod{31}$ has the
  two solutions
  \begin{equation*}
    x\equiv13\pmod{31}
    \quad\text{and}\quad
    x\equiv18\pmod{31}.
  \end{equation*}

  Finally, the congruence $x^2\equiv25\pmod{997}$ is easily seen to
  have solutions
  \begin{equation*}
    x\equiv5\pmod{997}
    \quad\text{and}\quad
    x\equiv992\pmod{997}. \qedhere
  \end{equation*}
\end{solution}

\Problem4 Find solutions for the congruences in
Problem~\ref{problem:quad-cong:sample2} that have them.
\begin{solution}
  Only $x^2\equiv625\pmod{9973}$ has a solution. Since $625 = 25^2$,
  we see that the two solutions to the congruence are
  \begin{equation*}
    x\equiv25\pmod{9973}
    \quad\text{and}\quad
    x\equiv9948\pmod{9973}. \qedhere
  \end{equation*}
\end{solution}

\Problem5 Calculate $(33/71)$, $(34/71)$, $(35/71)$, and $(36/71)$.
\begin{solution}
  Note that $71\equiv3\pmod4$ and $71\equiv7\pmod8$. Therefore
  \begin{align*}
    (33/71) &= (3/71)(11/71)
              = (71/3)(71/11) \\
            &= (2/3)(5/11)
              = (2/3)(11/5) \\
            &= (2/3)(1/5)
              = -1\cdot1 = -1, \\
    (34/71) &= (2/71)(17/71)
              = (71/17) \\
            &= (3/17)
              = (17/3) \\
            &= (2/3)
              = -1, \\
    (35/71) &= (5/71)(7/71)
              = -(71/5)(71/7) \\
            &= -(1/5)(1/7)
              = -1, \\
    \intertext{and}
    (36/71) &= 1. \qedhere
  \end{align*}
\end{solution}

\Problem6 Calculate $(33/73)$, $(34/73)$, $(35/73)$, and $(36/73)$.
\begin{solution}
  Note that $73\equiv1\pmod4$ and $73\equiv1\pmod8$. So
  \begin{align*}
    (33/73) &= (3/73)(11/73) \\
            &= (73/3)(73/11) \\
            &= (1/3)(7/11) \\
            &= -(11/7) = -(4/7) \\
            &= -1, \\
    (34/73) &= (2/73)(17/73) \\
            &= (73/17) = (5/17) \\
            &= (17/5) = (2/5) \\
            &= -1, \\
    (35/73) &= (5/73)(7/73) \\
            &= (73/5)(73/7) \\
            &= (3/5)(3/7) \\
            &= -(5/3)(7/3) \\
            &= -(2/3)(1/3)
              = 1, \\
    (36/73) &= 1. \qedhere
  \end{align*}
\end{solution}
