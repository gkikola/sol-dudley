\chapter{Fermat's and Wilson's Theorems}

\section{Exercises}

\Exercise1 Verify that Fermat's Theorem is true for $a = 2$ and $p = 5$.
\begin{solution}
  We have $a^{p-1} = 2^4 = 16 \equiv 1 \pmod 5$, so the theorem holds.
\end{solution}

\Exercise2 Calculate $2^2$ and $20^{10}\pmod{11}$.
\begin{solution}
  $2^2\equiv4\pmod{11}$. To find $20^{10}$, we note that
  $20^2 = 400 \equiv 4\pmod{11}$. Squaring gives
  $20^4\equiv16\equiv5\pmod{11}$. Squaring again gives
  $20^8\equiv25\equiv3\pmod{11}$. So
  \begin{equation*}
    20^{10} = 20^8\cdot20^2 \equiv 3\cdot4 \equiv 1 \pmod{11}.\qedhere
  \end{equation*}
\end{solution}

\Exercise3 In the proof of Wilson's Theorem, what are the pairs when
$p = 11$?
\begin{solution}
  To find the multiplicative inverse of $2$, we look for the least
  residue satisfying the congruence $2x\equiv1\pmod{11}$. Since
  $1\equiv12\pmod{11}$, we may divide by $2$ to get
  $x\equiv6\pmod{11}$. Hence $(2,6)$ is one such pair. In the same way
  we can find the remaining pairs. The complete list of pairs follows:
  \begin{equation*}
    (2,6), (3,4), (5,9), (7,8). \qedhere
  \end{equation*}
\end{solution}

\section{Problems}

\Problem1 What is the least residue of
\begin{equation*}
  5^6\pmod7 \qquad 5^8\pmod7 \qquad 1945^8\pmod7?
\end{equation*}
\begin{solution}
  $5^2 = 25 \equiv 4 \pmod7$, so $5^6 \equiv 4^3 = 64 \equiv 1\pmod7$.

  $5^8 = 5^6\cdot5^2 \equiv 1\cdot4 \equiv 4\pmod7$.

  For $1945^8$, note that $1945 = 5\cdot389$. Since
  $389\equiv4\pmod7$, we have
  \begin{equation*}
    1945^8 = 5^8\cdot389^8 \equiv 4\cdot4^8 \equiv 4^9\pmod7.
  \end{equation*}
  Now, $4^3\equiv 1\pmod7$, so $4^9\equiv1^3\equiv1\pmod7$. Therefore
  $1945^8\equiv1\pmod7$.
\end{solution}
