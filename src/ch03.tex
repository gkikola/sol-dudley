\chapter{Linear Diophantine Equations}

\section{Exercises}

\Exercise1 The equation $2x + 4y = 5$ has no solutions in
integers. Why not?
\begin{solution}
  If $x$ and $y$ are integers such that $2x + 4y = 5$, then
  $2(x + 2y) = 5$ and we see that $2\mid5$, which is clearly absurd.
\end{solution}

\Exercise2 Find by inspection a solution of $x + 5y = 10$ and use it
to write five other solutions.
\begin{solution}
  Certainly $x = 0$ and $y = 2$ works, so by Lemma~1 we also have the
  solutions
  \begin{equation*}
    x = 5t \quad\text{and}\quad y = 2 - t
  \end{equation*}
  for any integer $t$. Five such solutions, written as ordered pairs,
  are $(-10,4)$, $(-5,3)$, $(5,1)$, $(10,0)$, and $(15,-1)$.
\end{solution}

\Exercise3 Which of the following linear diophantine equations is
impossible? (We will say that a diophantine equation is {\em
  impossible} if it has no solutions).
\begin{enumerate}
\item $14x + 34y = 90$.
  \begin{solution}
    Since $(14,34) = 2$ and $2\mid90$, it follows by Lemma~2 that this
    equation has at least one solution.
  \end{solution}
\item $14x + 35y = 91$.
  \begin{solution}
    $(14,35) = 7$ and $7\mid91$, so this equation has a solution.
  \end{solution}
\item $14x + 36y = 93$.
  \begin{solution}
    This time, $(14,36) = 2$ but $2\nmid93$, so this equation is
    impossible.
  \end{solution}
\end{enumerate}

\Exercise4 Find all solutions of $2x + 6y = 20$.
\begin{solution}
  Dividing by $2$ gives $x + 3y = 10$. A particular solution is given
  by $(x_0,y_0) = (10,0)$, so by Lemma~3 all solutions have the form
  \begin{equation*}
    x = 10 + 3t \quad\text{and}\quad y = -t
  \end{equation*}
  where $t$ is an integer.
\end{solution}

\Exercise5 Find all the solutions of $2x + 6y = 18$ in {\em positive}
integers.
\begin{solution}
  In the text, the general solution was found to be
  \begin{equation*}
    x = 9 + 3t \quad\text{and}\quad y = -t,
  \end{equation*}
  for $t$ an integer. If $x$ is to be positive, then $9 + 3t > 0$ and,
  solving for $t$, we get $t > -3$. On the other hand, if $y > 0$ then
  $t < 0$. So we have $-3 < t < 0$ and we see that the only solutions
  are given by $t = -2$ and $t = -1$. These solutions are,
  respectively, $(3,2)$ and $(6,1)$.
\end{solution}

\section{Problems}

\Problem1 Find all the integer solutions of $x + y = 2$,
$3x - 4y = 5$, and $15x + 16y = 17$.
\label{problem:lin-dioph-eq:first-set}
\begin{solution}
  For $x + y = 2$, a particular solution is $(1,1)$, so the general
  solution is
  \begin{equation*}
    x = 1 + t \quad\text{and}\quad y = 1 - t,
  \end{equation*}
  where $t$ is an integer.

  For $3x - 4y = 5$ we find by inspection the particular solution
  $(3,1)$ which gives the general solution of
  \begin{equation*}
    x = 3 - 4t \quad\text{and}\quad y = 1 - 3t.
  \end{equation*}

  Lastly, for $15x + 16y = 17$, one solution is $(-1,2)$. Then the
  general solution is
  \begin{equation*}
    x = -1 + 16t
    \quad\text{and}\quad
    y = 2 - 15t. \qedhere
  \end{equation*}
\end{solution}

\Problem2 Find all the integer solutions of $2x + y = 2$,
$3x - 4y = 0$, and $15x + 18y = 17$.
\label{problem:lin-dioph-eq:second-set}
\begin{solution}
  For $2x + y = 2$, one solution is $(1,0)$, so the general solution
  is
  \begin{equation*}
    x = 1 + t \quad\text{and}\quad y = -2t
  \end{equation*}
  for an integer $t$.

  For $3x - 4y = 0$, a particular solution is $(4,3)$, producing the
  general solution
  \begin{equation*}
    x = 4 - 4t \quad\text{and}\quad y = 3 - 3t.
  \end{equation*}

  Lastly, the equation $15x + 18y = 17$ has no solutions since
  $(15,18) = 3$ but $3$ does not divide $17$.
\end{solution}

\Problem3 Find the solutions in positive integers of $x + y = 2$,
$3x - 4y = 5$, and $6x + 15y = 51$.
\begin{solution}
  In Problem~\ref{problem:lin-dioph-eq:first-set} we found the general
  solution of $x + y = 2$ to be $(1+t,1-t)$. If $x > 0$ then $t > -1$
  and if $y > 0$ then $t < 1$. So the only solution in positive
  integers is given by $t = 0$, which corresponds to the solution
  $(1,1)$.

  For $3x - 4y = 5$ we found the general solution to be
  $(3 - 4t, 1 - 3t)$. Setting $y > 0$ gives
  \begin{equation*}
    t < \frac13,
  \end{equation*}
  and we see that $x$ and $y$ are positive integers if and only if $t$
  is an integer with $t\leq0$. So the solutions are $(3,1)$, $(7,4)$,
  $(11,7)$, $\dots$.

  To solve $6x + 15y = 51$, we divide by $3$ to get $2x + 5y = 17$. A
  particular solution is $(1,3)$, leading to the general solution of
  $(1+5t,3-2t)$. By setting $x$ and $y$ greater than $0$, we determine
  that
  \begin{equation*}
    -\frac15 < t < \frac32.
  \end{equation*}
  So $t = 0$ or $1$, making the only positive solutions $(1,3)$ and
  $(6,1)$.
\end{solution}

\Problem4 Find all the solutions in positive integers of $2x + y = 2$,
$3x - 4y = 0$, and $7x + 15y = 51$.
\begin{solution}
  Using the results from
  Problem~\ref{problem:lin-dioph-eq:second-set}, the general solution
  for $2x + y = 2$ was $(1+t,-2t)$. Both variables are positive when
  $-1<t<0$, but there are no integers strictly between $-1$ and $0$,
  so there are no positive solutions.

  For $3x - 4y = 0$ we found the general solution $(4-4t,3-3t)$. All
  of these solutions are positive integers so long as
  $t\leq0$. Particular solutions are $(4,3)$, $(8,6)$, $(12,9)$, and
  so on.

  For $7x + 15y = 15$, we find the particular solution $(0,1)$ which
  leads to the general solution $(15t,1-7t)$. However, there is no
  integer value of $t$ which makes both $x$ and $y$ positive.
\end{solution}

\Problem5 Find all the positive solutions in integers of
\begin{align*}
  x + y + z &= 31, \\
  x + 2y + 3z &= 41.
\end{align*}
\begin{solution}
  Subtracting the first equation from the second gives
  \begin{equation*}
    y + 2z = 10.
  \end{equation*}
  This equation has the particular solution $(y,z) = (0,5)$ which
  leads to the general solution $(2t,5-t)$. Taking $y,z>0$ we find
  that the only relevant solutions are $(2,4)$, $(4,3)$, $(6,2)$, and
  $(8,1)$. Substituting these into either of the original equations
  allows us to find the corresponding values for $x$. The four
  solutions are
  \begin{gather*}
    x = 25,\quad y = 2,\quad\text{and}\quad z = 4; \\
    x = 24,\quad y = 4,\quad\text{and}\quad z = 3; \\
    x = 23,\quad y = 6,\quad\text{and}\quad z = 2; \\
    x = 22,\quad y = 8,\quad\text{and}\quad z = 1. \qedhere
  \end{gather*}
\end{solution}

\Problem6 Find the five different ways a collection of $100$
coins---pennies, dimes, and quarters---can be worth exactly $\$4.99$.
\begin{solution}
  Let $x$ be the number of pennies, $y$ the number of dimes, and $z$
  the number of quarters. Since there are $100$ coins, whose total
  value is $\$4.99$, we have the two equations
  \begin{align*}
    x + y + z &= 100 \\
    x + 10y + 25z &= 499.
  \end{align*}

  Subtracting the first equation from the second gives
  $9y + 24z = 399$. Dividing this equation by $3$ then gives
  $3y + 8z = 133$. By inspection, a particular solution is $y = 7$ and
  $z = 14$. This gives the general solution $y = 7 + 8t$ and
  $z = 14 - 3t$. We find the positive solutions to be
  \begin{align*}
    t &= 0\colon\quad x = 79, \quad y = 7, \quad\text{and}\quad z = 14; \\
    t &= 1\colon\quad x = 74, \quad y = 15, \quad\text{and}\quad z = 11; \\
    t &= 2\colon\quad x = 69, \quad y = 23, \quad\text{and}\quad z = 8; \\
    t &= 3\colon\quad x = 64, \quad y = 31, \quad\text{and}\quad z = 5; \\
    t &= 4\colon\quad x = 59, \quad y = 39, \quad\text{and}\quad z = 2.
  \end{align*}
  These are the only five solutions in the positive integers.
\end{solution}

\Problem7 A man bought a dozen pieces of fruit---apples and
oranges---for $99$ cents. If an apple costs $3$ cents more than an
orange, and he bought more apples than oranges, how many of each did
he buy?
\begin{solution}
  Let $x$ be the number of apples that the man bought, and let $y$ be
  the number of oranges. The equation $x + y = 12$ has only five
  solutions in the positive integers with $x > y$, namely $(7,5)$,
  $(8,4)$, $(9,3)$, $(10,2)$, and $(11,1)$.

  Now, if $a$ is the price of an apple, then the solution for $x$ and
  $y$ must also satisfy the equation $ax + (a-3)y = 99$. If we
  substitute the solution $(7,5)$ into this equation and simplify, we
  get $12a - 15 = 99$ or $a = 19/2$, which is not an
  integer. Similarly, the solutions $(8,4)$, $(10,2)$, and $(11,1)$
  also lead to non-integer values of $a$. The only solution that works
  is
  \begin{equation*}
    x = 9 \quad\text{and}\quad y = 3,
  \end{equation*}
  with $a = 9$. Therefore, the man bought $9$ apples at $9$ cents
  each, and $3$ oranges at $6$ cents each.
\end{solution}

\Problem8 The enrollment in a number theory class consists of
sophomores, juniors, and backward seniors. If each sophomore
contributes $\$1.25$, each junior $\$.90$, and each senior $\$.50$,
the instructor will have a fund of $\$25$. There are $26$ students;
how many of each?
\begin{solution}
  Let $x$ be the number of sophomores, $y$ the number of juniors, and
  $z$ the number of seniors. Then we have the following system of
  equations:
  \begin{align}
    \label{eq:lin-dioph-eq:student-counts}
    x + y + z &= 26, \\
    \label{eq:lin-dioph-eq:student-costs}
    125x + 90y + 50z &= 2500.
  \end{align}
  Multiplying \eqref{eq:lin-dioph-eq:student-counts} by $50$ and
  subtracting from \eqref{eq:lin-dioph-eq:student-costs} gives the
  equation
  \begin{equation*}
    75x + 40y = 1200.
  \end{equation*}
  Dividing by $5$ gives $15x + 8y = 240$. A particular solution is
  $(0,30)$, so we have the general solution
  \begin{equation*}
    x = 8t \quad\text{and}\quad y = 30 - 15t.
  \end{equation*}
  If $x$ and $y$ are to be positive, we see that $0 < t < 2$, so that
  $t = 1$. Therefore, there are $8$ sophomores, $15$ juniors, and $3$
  seniors.
\end{solution}

\Problem9 The following problem first appeared in an Indian book
written around $850$ AD. Three merchants found a purse along the
way. One of them said, ``If I secure this purse, I shall become twice
as rich as both of you with your money on hand.'' Then the second
said, ``I shall become thrice as rich as both of you.'' The third man
said, ``I shall become five times as rich as both of you.'' How much
did each merchant have, and how much was in the purse?
\begin{solution}
  Let the three merchants each have $x$, $y$, and $z$ units of
  currency, respectively, and let $w$ be the amount of money in the
  purse. We have the following system of equations.
  \begin{align*}
    x + w &= 2(y + z), \\
    y + w &= 3(x + z), \\
    z + w &= 5(x + y).
  \end{align*}
  Rearranging and simplifying then gives
  \begin{align*}
    x - 2y - 2z + w &= 0, \\
    -3x + y - 3z + w &= 0, \\
    -5x -5y + z + w &= 0.
  \end{align*}
  Solving these simultaneously, we find that the system reduces to
  \begin{align*}
    15x - w &= 0, \\
    5y - w &= 0, \\
    3z - w &= 0.
  \end{align*}
  So the purse has $15$ times as much money as the first merchant, the
  second merchant has $3$ times as much money as the first merchant,
  and the third merchant has $5$ times as much money as the first
  merchant. So the three merchants and the purse have, respectively,
  $x$, $3x$, $5x$, and $15x$ units of currency, for an integer
  $x$. Any positive value for $x$ will produce a valid solution.
\end{solution}

\Problem{10} A man cashes a check for $d$ dollars and $c$ cents at a
bank. Assume that the teller by mistake gives the man $c$ dollars and
$d$ cents. Assume that the man does not notice the error until he has
spent $23$ cents. Assume further that he then notices that he has $2d$
dollars and $2c$ cents. Assume still further that he asks you what
amount the check was for. Assuming that you can accept all the
assumptions, what is the answer?
\begin{solution}
  Let the check be for $T$ cents. The man starts with $c$ dollars and
  $d$ cents. After spending $23$ cents, he has $2d$ dollars and $2c$
  cents. This gives
  \begin{align*}
    100d + c &= T, \\
    100c + d - 23 &= 100(2d) + 2c,
  \end{align*}
  or, rearranging,
  \begin{align*}
    c + 100d &= T, \\
    98c - 199d &= 23.
  \end{align*}
  By inspection, a particular solution to $98c - 199d = 23$ is
  $c = 51$ and $d = 25$. The general solution is then
  \begin{equation*}
    c = 51 + 199t \quad\text{and}\quad d = 25 + 98t.
  \end{equation*}
  We know $0\leq c<100$ so the only possible value for $t$ is $t =
  0$. Therefore the check was written for $T = 25\cdot100 + 51 = 2551$
  cents or $\$25.51$.
\end{solution}
