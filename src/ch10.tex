\chapter{Primitive Roots}

\section{Exercises}

\Exercise1 What are the orders of $3$, $5$, and $7$, modulo $8$?
\begin{solution}
  Since $3^2\equiv1\pmod8$, $5^2\equiv1\pmod8$, and
  $7^2\equiv1\pmod8$, we see that each of these residues has order
  $2$.
\end{solution}

\Exercise2 What order can an integer have (mod $9$)? Find an example
of each.
\begin{solution}
  Since $\phi(9) = 3\cdot2 = 6$, the possible orders are $1$, $2$,
  $3$, and $6$. $1$ is the only integer with order $1$. $8$ has order
  $2$ since $8^2\equiv 1\pmod9$. $4$ has order $3$ since
  $4^2 \equiv 7\pmod9$ and $4^3\equiv1\pmod9$. And $2$ has order $6$,
  since $2^2\equiv4\pmod9$ and $2^3\equiv8\pmod9$.
\end{solution}

\Exercise3 Using the corollary to Theorem~3, what is the smallest
possible prime divisor of $2^{19} - 1$?
\begin{solution}
  By the corollary, any divisor can be written
  $2\cdot19k + 1 = 38k + 1$. The smallest numbers of this form are
  $1$, $39$, $77$, $115$, $153$, and $191$. The smallest possible
  prime divisor is therefore $191$. ($191$ is not a divisor, however,
  as $2^{19} - 1$ happens to be prime).
\end{solution}
