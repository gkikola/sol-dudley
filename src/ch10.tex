\chapter{Primitive Roots}

\section{Exercises}

\Exercise1 What are the orders of $3$, $5$, and $7$, modulo $8$?
\begin{solution}
  Since $3^2\equiv1\pmod8$, $5^2\equiv1\pmod8$, and
  $7^2\equiv1\pmod8$, we see that each of these residues has order
  $2$.
\end{solution}

\Exercise2 What order can an integer have (mod $9$)? Find an example
of each.
\begin{solution}
  Since $\phi(9) = 3\cdot2 = 6$, the possible orders are $1$, $2$,
  $3$, and $6$. $1$ is the only integer with order $1$. $8$ has order
  $2$ since $8^2\equiv 1\pmod9$. $4$ has order $3$ since
  $4^2 \equiv 7\pmod9$ and $4^3\equiv1\pmod9$. And $2$ has order $6$,
  since $2^2\equiv4\pmod9$ and $2^3\equiv8\pmod9$.
\end{solution}

\Exercise3 Using the corollary to Theorem~3, what is the smallest
possible prime divisor of $2^{19} - 1$?
\begin{solution}
  By the corollary, any divisor can be written
  $2\cdot19k + 1 = 38k + 1$. The smallest numbers of this form are
  $1$, $39$, $77$, $115$, $153$, and $191$. The smallest possible
  prime divisor is therefore $191$. ($191$ is not a divisor, however,
  as $2^{19} - 1$ happens to be prime).
\end{solution}

\Exercise4 Show that $3$ is a primitive root of $7$.
\begin{solution}
  We get
  \begin{align*}
    3^1 &\equiv 3\pmod7, \\
    3^2 &\equiv 2\pmod7, \\
    3^3 &\equiv 6\pmod7, \\
    3^4 &\equiv 4\pmod7, \\
    3^5 &\equiv 5\pmod7, \\
    3^6 &\equiv 1\pmod7.
  \end{align*}
  Since $\phi(7) = 6$, $3$ is a primitive root of $7$.
\end{solution}

\Exercise5 Find, by trial, a primitive root of $10$.
\begin{solution}
  $\phi(10) = 4$. The powers of $3$ (mod $10$) are, respectively, $3$,
  $9$, $7$, and $1$, so $3$ is a primitive root of $10$. $7$ is also a
  primitive root.
\end{solution}

\Exercise6 Use the table of powers (mod $11$) at the beginning of this
section to verify that the corollary is true for $p = 11$.
\begin{solution}
  There is only one least residue with order $2$ (namely $10$), and
  $\phi(2) = 1$. There are $4$ residues with order $5$ ($3$, $4$, $5$,
  and $9$), and $\phi(5) = 4$. And there are $4$ residues with order
  $10$ ($2$, $6$, $7$, and $8$), and $\phi(10) = 4$. In each case the
  corollary holds.
\end{solution}

\Exercise7 Which of the integers $2, 3, \dots, 25$ do not have
primitive roots?
\begin{solution}
  The only integers with primitive roots are $1$, $2$, $4$, $p^e$, and
  $2p^e$ where $p$ is an odd prime. So $8$, $12$, $15$, $16$, $20$,
  $21$, and $24$ do not have primitive roots.
\end{solution}

\section{Problems}

\Problem1 Find the orders of $1, 2, \dots, 12$ (mod $13$).
\begin{solution}
  $1$ has order $1$. $12$ has order $2$. $3$ and $9$ have order
  $3$. $5$ and $8$ have order $4$. $4$ and $10$ have order $6$. $2$,
  $6$, $7$, and $11$ have order $12$.
\end{solution}

\Problem2 Find the orders of $1, 2, \dots, 16$ (mod $17$).
\begin{solution}
  $1$ has order $1$. $16$ has order $2$. $4$ and $13$ have order
  $4$. $2$, $8$, $9$, and $15$ have order $8$. $3$, $5$, $6$, $7$,
  $10$, $11$, $12$, and $14$ have order $16$.
\end{solution}

\Problem3 One of the primitive roots of $19$ is $2$. Find all of the
others.
\begin{solution}
  According to the corollary to Lemma~1, $2^k$ will be a primitive
  root of $19$ when $(k,18) = 1$. So the primitive roots are $2$,
  $2^5\equiv13$, $2^7\equiv14$, $2^{11}\equiv15$, $2^{13}\equiv3$, and
  $2^{17}\equiv10\pmod{19}$.
\end{solution}

\Problem4 One of the primitive roots of $23$ is $5$. Find all of the
others.
\begin{solution}
  $5^k$ should be a primitive root when $(k,22) = 1$. So the primitive
  roots are $5$, $5^3$, $5^5$, $5^7$, $5^9$, $5^{13}$, $5^{15}$,
  $5^{17}$, $5^{19}$, and $5^{21}$. Computing these powers, we find
  that the primitive roots are $7$, $10$, $11$, $14$, $15$, $17$,
  $19$, $20$, and $21$.
\end{solution}

\Problem5 What are the orders of $2$, $4$, $7$, $8$, $11$, $13$, and
$14$ (mod $15$)? Does $15$ have primitive roots?
\begin{solution}
  $4$, $11$, and $14$ have order $2$. $2$, $7$, $8$, and $13$ have
  order $4$. No integer has an order of $\phi(15) = 8$, so $15$ does
  not have primitive roots.
\end{solution}

\Problem6 What are the orders of $3$, $7$, $9$, $11$, $13$, $17$, and
$19$ (mod $20$)? Does $20$ have primitive roots?
\begin{solution}
  $9$, $11$, and $19$ have order $2$. $3$, $7$, $13$, and $17$ have
  order $4$. No integer has an order of $\phi(20) = 8$, so $20$ does
  not have primitive roots.
\end{solution}

\Problem7 Which integers have order $6$ (mod $31$)?
\begin{solution}
  By inspection, $6$ has order $6$ (mod $31$). As in the proof of
  Theorem~6, we can find the other integers with order $6$ by taking
  $6^k$ for $k$ such that $(k,6) = 1$. So the least residues having
  order $6$ are $6$ and $6^5\equiv26\pmod{31}$.
\end{solution}

\Problem8 Which integers have order $6$ (mod $37$)?
\begin{solution}
  $11$ has order $6$, so $11^5\equiv27\pmod{37}$ also has order $6$.
\end{solution}
