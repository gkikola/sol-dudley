\chapter{Perfect Numbers}

\section{Exercises}

\Exercise1 Verify that $1184$ and $1210$ are amicable.
\begin{solution}
  $1184 = 2^5\cdot37$ and $1210 = 2\cdot5\cdot11^2$, so
  \begin{equation*}
    \sigma(1184) = \sigma(2^5)\sigma(37)
    = (2^6 - 1)(38) = 63\cdot38 = 2394
  \end{equation*}
  and
  \begin{equation*}
    \sigma(1210) = \sigma(2)\sigma(5)\sigma(11^2)
    = 3\cdot6\cdot(1 + 11 + 121) = 18\cdot133
    = 2394.
  \end{equation*}
  Since $1184 + 1210 = 2394$, we see that the two numbers form an
  amicable pair.
\end{solution}

\section{Problems}

\Problem1 Verify that $2620, 2924$ and $17296,18416$ are amicable pairs.
\begin{solution}
  $2620 = 2^2\cdot5\cdot131$ and $2924 = 2^2\cdot17\cdot43$. We get
  \begin{equation*}
    \sigma(2620) = 7\cdot6\cdot132 = 5544
  \end{equation*}
  and
  \begin{equation*}
    \sigma(2924) = 7\cdot18\cdot44 = 5544,
  \end{equation*}
  and $2620 + 2924 = 5544$, so the two are amicable.

  For $17296$ we have
  \begin{equation*}
    \sigma(17296) = \sigma(2^4\cdot23\cdot47)
    = 31\cdot24\cdot48 = 35712
  \end{equation*}
  and for $18416$ we have
  \begin{equation*}
    \sigma(18416) = \sigma(2^4\cdot1151)
    = 31\cdot1152 = 35712,
  \end{equation*}
  and $17296 + 18416 = 35712$, so these two are also amicable.
\end{solution}

\Problem2 It was long thought that even perfect numbers ended
alternately in $6$ and $8$. Show that this is wrong by verifying that
the perfect numbers corresponding to the primes $2^{13} - 1$ and
$2^{17} - 1$ both end in $6$.
\begin{proof}
  First note that $2^8 = 256\equiv6\pmod{10}$ so
  $2^{16}\equiv6^2\equiv6\pmod{10}$, and
  $2^{12} = 2^8\cdot2^4\equiv6\cdot6\equiv6\pmod{10}$.

  For $p = 13$, we have
  \begin{equation*}
    2^{p-1}(2^p - 1)
    = 2^{12}(2^{13} - 1)
    \equiv 6\cdot(6\cdot2 - 1)
    \equiv6\pmod{10}
  \end{equation*}
  and for $p = 17$ we have
  \begin{equation*}
    2^{16}(2^{17} - 1)
    = 6\cdot(6\cdot2 - 1)
    \equiv6\pmod{10}.
  \end{equation*}
  In both cases, we see that the corresponding perfect numbers end in
  $6$.
\end{proof}
