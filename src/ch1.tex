\chapter{Integers}

\Exercise1 Which integers divide zero?
\begin{solution}
  Every integer divides $0$. For, if $k$ is any integer, then $0k = 0$
  so that $k\mid0$.
\end{solution}

\Exercise2 Show that if $a\mid b$ and $b\mid c$ then, $a\mid c$.
\begin{proof}
  Let $a\mid b$ and $b\mid c$. Then there are integers $m$ and $n$
  such that $am = b$ and $bn = c$. But then $a(mn) = (am)n = bn =
  c$. Since $mn$ is an integer, we have $a\mid c$.
\end{proof}

\Exercise3 Prove that if $d\mid a$ then $d\mid ca$ for any integer $c$.
\begin{proof}
  Again, by definition we can find an integer $n$ such that $dn =
  a$. But then $cdn = ca$. Since $cn$ is an integer, it follows that
  $d\mid ca$.
\end{proof}

\Exercise4 What are $(4,14)$, $(5,15)$, and $(6,16)$?
\begin{solution}
  By inspection, $(4,14) = 2$, $(5,15) = 5$, and $(6,16) = 2$.
\end{solution}

\Exercise5 What is $(n,1)$, where $n$ is any positive integer? What is
$(n,0)$?
\begin{solution}
  We have $(n,1) = 1$ since there is no integer greater than $1$ which
  divides $1$. We also have $(n,0) = n$ since no integer larger than
  $n$ can divide $n$, and $n$ certainly divides itself and $0$.
\end{solution}

\Exercise6 If $d$ is a positive integer, what is $(d, nd)$?
\begin{solution}
  $(d,nd) = d$ since $d$ is a common divisor ($d\mid nd$ by Lemma~2)
  and there can be no greater divisor of $d$.
\end{solution}

\Exercise7 What are $q$ and $r$ if $a = 75$ and $b = 24$? If $a = 75$
and $b = 25$?
\begin{solution}
  We have
  \begin{equation*}
    75 = 3(24) + 3
    \quad\text{and}\quad
    75 = 3(25) + 0.
  \end{equation*}
  So $q = 3$ and $r = 3$ in the first case, and $q = 3$ and $r = 0$ in
  the second.
\end{solution}

\Exercise8 Verify that Lemma~3 is true when $a = 16$, $b = 6$, and
$q = 2$.
\begin{solution}
  Since $16 = 6\cdot2 + 4$, we have $r = 4$. And since
  $(16,6) = 2 = (6,4)$, the lemma is true for this case.
\end{solution}

\Exercise9 Calculate $(343, 280)$ and $(578, 442)$.
\begin{solution}
  Following the Euclidean Algorithm, we have
  \begin{align*}
    343 &= 280\cdot1 + 63 \\
    280 &= 63\cdot4 + 28 \\
    63 &= 28\cdot2 + 7 \\
    28 &= 7\cdot4.
  \end{align*}
  Therefore $(343, 280) = 7$.

  For the second pair,
  \begin{align*}
    578 &= 442\cdot1 + 136 \\
    442 &= 136\cdot3 + 34 \\
    136 &= 34\cdot4,
  \end{align*}
  so $(578,442) = 34$.
\end{solution}
