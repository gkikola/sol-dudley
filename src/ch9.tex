\chapter{Euler's Theorem and Function}

\section{Exercises}

\Exercise1 Show that $a^6\equiv1\pmod{14}$ for all $a$ relatively
prime to $14$.
\begin{solution}
  The least residues that are relatively prime to $14$ are $1$, $3$,
  $5$, $9$, $11$, and $13$. We compute:
  \begin{align*}
    3^6 &= 27^2 \equiv (-1)^2 \equiv 1\pmod{14}, \\
    5^6 &= 25^3 \equiv (-3)^3 \equiv 1\pmod{14}, \\
    9^6 &= (3^6)^2 \equiv 1\pmod{14}, \\
    11^6 &\equiv (-3)^6 \equiv 1\pmod{14}, \\
    \intertext{and}
    13^6 &\equiv (-1)^6 \equiv 1\pmod{14}.
  \end{align*}
  In every case, $(a,14) = 1$ implies $a^6\equiv1\pmod{14}$.
\end{solution}

\Exercise2 Verify that Lemma~1 is true if $m = 14$ and $a = 5$.
\begin{solution}
  Again, the relatively prime positive integers less than $14$ are
  $1$, $3$, $5$, $9$, $11$, and $13$.
  \begin{align*}
    5\cdot1 &= 5 \equiv 5\pmod{14}, \\
    5\cdot3 &= 15 \equiv 1\pmod{14}, \\
    5\cdot5 &= 25 \equiv 11\pmod{14}, \\
    5\cdot9 &= 45 \equiv 3\pmod{14}, \\
    5\cdot11 &= 55 \equiv 13\pmod{14}, \\
    \intertext{and}
    5\cdot13 &= 65 \equiv 9\pmod{14}.
  \end{align*}
  And, certainly, $(5,1,11,3,13,9)$ is a permutation of
  $\{1,3,5,9,11,13\}$.
\end{solution}
