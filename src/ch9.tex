\chapter{Euler's Theorem and Function}

\section{Exercises}

\Exercise1 Show that $a^6\equiv1\pmod{14}$ for all $a$ relatively
prime to $14$.
\begin{solution}
  The least residues that are relatively prime to $14$ are $1$, $3$,
  $5$, $9$, $11$, and $13$. We compute:
  \begin{align*}
    3^6 &= 27^2 \equiv (-1)^2 \equiv 1\pmod{14}, \\
    5^6 &= 25^3 \equiv (-3)^3 \equiv 1\pmod{14}, \\
    9^6 &= (3^6)^2 \equiv 1\pmod{14}, \\
    11^6 &\equiv (-3)^6 \equiv 1\pmod{14}, \\
    \intertext{and}
    13^6 &\equiv (-1)^6 \equiv 1\pmod{14}.
  \end{align*}
  In every case, $(a,14) = 1$ implies $a^6\equiv1\pmod{14}$.
\end{solution}

\Exercise2 Verify that Lemma~1 is true if $m = 14$ and $a = 5$.
\begin{solution}
  Again, the relatively prime positive integers less than $14$ are
  $1$, $3$, $5$, $9$, $11$, and $13$.
  \begin{align*}
    5\cdot1 &= 5 \equiv 5\pmod{14}, \\
    5\cdot3 &= 15 \equiv 1\pmod{14}, \\
    5\cdot5 &= 25 \equiv 11\pmod{14}, \\
    5\cdot9 &= 45 \equiv 3\pmod{14}, \\
    5\cdot11 &= 55 \equiv 13\pmod{14}, \\
    \intertext{and}
    5\cdot13 &= 65 \equiv 9\pmod{14}.
  \end{align*}
  And, certainly, $(5,1,11,3,13,9)$ is a permutation of
  $\{1,3,5,9,11,13\}$.
\end{solution}

\Exercise3 Verify that the entries in the following table are correct.
\begin{center}
  \begin{tabular}{r|rrrrrrrrr}
    $n$ & 2 & 3 & 4 & 5 & 6 & 7 & 8 & 9 & 10 \\
    $\phi(n)$ & 1 & 2 & 2 & 4 & 2 & 6 & 4 & 6 & 4
  \end{tabular}
\end{center}
\begin{solution}
  The verification is straightforward. We simply count the number of
  positive integers that are relatively prime to $n$ and less than or
  equal to $n$. For example, when $n = 8$ the relatively prime
  residues are $1$, $3$, $5$, and $7$, so $\phi(8) = 4$. The other
  values are checked in the same way.
\end{solution}

\Exercise4 Verify that $3^{\phi(8)}\equiv1\pmod8$.
\begin{solution}
  Since $\phi(8) = 4$, we have $3^4 = 9^2 \equiv 1^2 \equiv 1\pmod8$.
\end{solution}

\Exercise5 Which positive integers are less than $4$ and relatively
prime to it? What is the answer if $4$ is replaced by $8$? By $16$?
Can you induce a formula for $\phi(2^n)$, $n = 1,2,\dots$?
\begin{solution}
  For $4$ the numbers are $1$ and $3$. For $8$ they are $1$, $3$, $5$,
  and $7$. For $16$, they are $1$, $3$, $5$, $7$, $9$, $11$, $13$, and
  $15$. So we have $\phi(4) = 2$, $\phi(8) = 4$, and $\phi(16) = 8$.

  In general, it looks like $\phi(2^n) = 2^{n-1}$ for each positive
  $n$. To prove this, note that there are $2^n$ positive integers less
  than or equal to $2^n$. Exactly half of these numbers will be even
  and thus not relatively prime to $2^n$. So
  \begin{equation*}
    \phi(2^n) = \frac12(2^n) = 2^{n-1}. \qedhere
  \end{equation*}
\end{solution}

\Exercise6 Verify that the formula of Lemma~2 is correct for $p = 5$
and $n = 2$.
\begin{solution}
  According to Lemma~2, $\phi(5^2) = 5^1(5 - 1) = 20$. The positive
  integers less than or equal to $5^2$ that are not relatively prime
  to it are $5$, $10$, $15$, $20$, and $25$, so
  $\phi(5^2) = 25 - 5 = 20$ and the formula works in this case.
\end{solution}

\Exercise7 In the proof of Theorem~2, how many rows are there whose
first element is relatively prime to $m$?
\begin{solution}
  There are exactly $\phi(m)$ such rows.
\end{solution}

\Exercise8 Calculate $\phi(74)$, $\phi(76)$, and $\phi(78)$.
\begin{solution}
  Using Theorem~3, we get
  \begin{align*}
    \phi(74) &= \phi(2)\phi(37) = (1\cdot1)(1\cdot36) = 36, \\
    \phi(76) &= \phi(2^2)\phi(19)
               = (2\cdot1)(1\cdot18) = 36, \\
    \intertext{and}
    \phi(78) &= \phi(2)\phi(3)\phi(13)
               = (1\cdot1)(1\cdot2)(1\cdot12) = 24. \qedhere
  \end{align*}
\end{solution}
