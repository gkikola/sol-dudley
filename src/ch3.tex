\chapter{Linear Diophantine Equations}

\section{Exercises}

\Exercise1 The equation $2x + 4y = 5$ has no solutions in
integers. Why not?
\begin{solution}
  If $x$ and $y$ are integers such that $2x + 4y = 5$, then
  $2(x + 2y) = 5$ and we see that $2\mid5$, which is clearly absurd.
\end{solution}

\Exercise2 Find by inspection a solution of $x + 5y = 10$ and use it
to write five other solutions.
\begin{solution}
  Certainly $x = 0$ and $y = 2$ works, so by Lemma~1 we also have the
  solutions
  \begin{equation*}
    x = 5t \quad\text{and}\quad y = 2 - t
  \end{equation*}
  for any integer $t$. Five such solutions, written as ordered pairs,
  are $(-10,4)$, $(-5,3)$, $(5,1)$, $(10,0)$, and $(15,-1)$.
\end{solution}

\Exercise3 Which of the following linear diophantine equations is
impossible? (We will say that a diophantine equation is {\em
  impossible} if it has no solutions).
\begin{enumerate}
\item $14x + 34y = 90$.
  \begin{solution}
    Since $(14,34) = 2$ and $2\mid90$, it follows by Lemma~2 that this
    equation has at least one solution.
  \end{solution}
\item $14x + 35y = 91$.
  \begin{solution}
    $(14,35) = 7$ and $7\mid91$, so this equation has a solution.
  \end{solution}
\item $14x + 36y = 93$.
  \begin{solution}
    This time, $(14,36) = 2$ but $2\nmid93$, so this equation is
    impossible.
  \end{solution}
\end{enumerate}

\Exercise4 Find all solutions of $2x + 6y = 20$.
\begin{solution}
  Dividing by $2$ gives $x + 3y = 10$. A particular solution is given
  by $(x_0,y_0) = (10,0)$, so by Lemma~3 all solutions have the form
  \begin{equation*}
    x = 10 + 3t \quad\text{and}\quad y = -t
  \end{equation*}
  where $t$ is an integer.
\end{solution}

\Exercise5 Find all the solutions of $2x + 6y = 18$ in {\em positive}
integers.
\begin{solution}
  In the text, the general solution was found to be
  \begin{equation*}
    x = 9 + 3t \quad\text{and}\quad y = -t,
  \end{equation*}
  for $t$ an integer. If $x$ is to be positive, then $9 + 3t > 0$ and,
  solving for $t$, we get $t > -3$. On the other hand, if $y > 0$ then
  $t < 0$. So we have $-3 < t < 0$ and we see that the only solutions
  are given by $t = -2$ and $t = -1$. These solutions are,
  respectively, $(3,2)$ and $(6,1)$.
\end{solution}

\section{Problems}

\Problem1 Find all the integer solutions of $x + y = 2$,
$3x - 4y = 5$, and $15x + 16y = 17$.
\label{problem:lin-dioph-eq:first-set}
\begin{solution}
  For $x + y = 2$, a particular solution is $(1,1)$, so the general
  solution is
  \begin{equation*}
    x = 1 + t \quad\text{and}\quad y = 1 - t,
  \end{equation*}
  where $t$ is an integer.

  For $3x - 4y = 5$ we find by inspection the particular solution
  $(3,1)$ which gives the general solution of
  \begin{equation*}
    x = 3 - 4t \quad\text{and}\quad y = 1 - 3t.
  \end{equation*}

  Lastly, for $15x + 16y = 17$, one solution is $(-1,2)$. Then the
  general solution is
  \begin{equation*}
    x = -1 + 16t
    \quad\text{and}\quad
    y = 2 - 15t. \qedhere
  \end{equation*}
\end{solution}

\Problem2 Find all the integer solutions of $2x + y = 2$,
$3x - 4y = 0$, and $15x + 18y = 17$.
\label{problem:lin-dioph-eq:second-set}
\begin{solution}
  For $2x + y = 2$, one solution is $(1,0)$, so the general solution
  is
  \begin{equation*}
    x = 1 + t \quad\text{and}\quad y = -2t
  \end{equation*}
  for an integer $t$.

  For $3x - 4y = 0$, a particular solution is $(4,3)$, producing the
  general solution
  \begin{equation*}
    x = 4 - 4t \quad\text{and}\quad y = 3 - 3t.
  \end{equation*}

  Lastly, the equation $15x + 18y = 17$ has no solutions since
  $(15,18) = 3$ but $3$ does not divide $17$.
\end{solution}

\Problem3 Find the solutions in positive integers of $x + y = 2$,
$3x - 4y = 5$, and $6x + 15y = 51$.
\begin{solution}
  In Problem~\ref{problem:lin-dioph-eq:first-set} we found the general
  solution of $x + y = 2$ to be $(1+t,1-t)$. If $x > 0$ then $t > -1$
  and if $y > 0$ then $t < 1$. So the only solution in positive
  integers is given by $t = 0$, which corresponds to the solution
  $(1,1)$.

  For $3x - 4y = 5$ we found the general solution to be
  $(3 - 4t, 1 - 3t)$. Setting $y > 0$ gives
  \begin{equation*}
    t < \frac13,
  \end{equation*}
  and we see that $x$ and $y$ are positive integers if and only if $t$
  is an integer with $t\leq0$. So the solutions are $(3,1)$, $(7,4)$,
  $(11,7)$, $\dots$.

  To solve $6x + 15y = 51$, we divide by $3$ to get $2x + 5y = 17$. A
  particular solution is $(1,3)$, leading to the general solution of
  $(1+5t,3-2t)$. By setting $x$ and $y$ greater than $0$, we determine
  that
  \begin{equation*}
    -\frac15 < t < \frac32.
  \end{equation*}
  So $t = 0$ or $1$, making the only positive solutions $(1,3)$ and
  $(6,1)$.
\end{solution}

\Problem4 Find all the solutions in positive integers of $2x + y = 2$,
$3x - 4y = 0$, and $7x + 15y = 51$.
\begin{solution}
  Using the results from
  Problem~\ref{problem:lin-dioph-eq:second-set}, the general solution
  for $2x + y = 2$ was $(1+t,-2t)$. Both variables are positive when
  $-1<t<0$, but there are no integers strictly between $-1$ and $0$,
  so there are no positive solutions.

  For $3x - 4y = 0$ we found the general solution $(4-4t,3-3t)$. All
  of these solutions are positive integers so long as
  $t\leq0$. Particular solutions are $(4,3)$, $(8,6)$, $(12,9)$, and
  so on.

  For $7x + 15y = 15$, we find the particular solution $(0,1)$ which
  leads to the general solution $(15t,1-7t)$. However, there is no
  integer value of $t$ which makes both $x$ and $y$ positive.
\end{solution}
