\chapter{Unique Factorization}

\section{Exercises}

\Exercise1 How many even primes are there? How many whose last digit
is $5$?
\begin{solution}
  If a prime $p$ is even then by definition $2\mid p$. Therefore the
  only prime that is even is $2$ itself. Similarly, any positive
  integer that ends in a $5$ (written in base $10$) must be divisible
  by $5$ (this is due to the fact that our base, $10$, is itself
  divisible by $5$). And the only prime divisible by $5$ is $5$
  itself.
\end{solution}

\Exercise2 Construct a proof of Lemma~$2$ using induction.
\begin{solution}
  Lemma~$2$ says that every positive integer greater than $1$ can be
  written as a product of primes. $2$ is a prime and is a product of
  itself, so the base case is satisfied. Now suppose there is an
  integer $n>1$ such that every integer $k$ with $1<k\leq n$ can be
  written as a product of primes. We must show that $n+1$ can be
  written as such a product.

  If $n+1$ is prime, then we are done, it is already a product of
  primes. If not, then $n+1$ is composite, and we may write $n+1 = st$
  where $s$ and $t$ are each integers with $1<s,t<n+1$. By the
  inductive hypothesis, $s$ and $t$ can each be written as a product
  of primes,
  \begin{equation*}
    s = p_1p_2\cdots p_i,
    \quad\text{and}\quad
    t = q_1q_2\cdots q_j,
  \end{equation*}
  where each $p_k$ and $q_k$ are prime (not necessarily
  distinct). Then
  \begin{equation*}
    n+1 = st = p_1p_2\cdots p_iq_1q_2\cdots q_j,
  \end{equation*}
  and we have written $n+1$ as a product of primes, completing the
  inductive step. It follows by induction that all integers $n>1$ can
  be written as a product of primes.
\end{solution}

\Exercise3 Write prime decompositions for $72$ and $480$.
\begin{solution}
  $72 = 8\cdot9 = 2^3\cdot3^2$ and
  $480 = 48\cdot10 = 16\cdot3\cdot10 = 2^5\cdot3\cdot5$.
\end{solution}

\Exercise4 Which members of the set less than $100$ are not prome?
\begin{solution}
  The set being referenced in the question is the set
  \begin{equation*}
    A = \{4n+1\mid n = 0, 1, 2, \dots\},
  \end{equation*}
  where $k\in A$ is considered ``prome'' if it has no divisors in $A$
  other than $1$ and itself.

  Since $100^{1/2} = 10$, we only need to look for divisors less than
  or equal to $10$. The only such members of $A$ are $1$, $5$, and
  $9$. So any nonprome member of $A$ less than $100$ must be a
  multiple of $5$ or $9$. These numbers are
  \begin{equation*}
    25, 45, 65, 81, 85. \qedhere
  \end{equation*}
\end{solution}

\Exercise5 What is the prime-power decomposition of $7950$?
\begin{solution}
  $7950$ is divisible by $50 = 2\cdot5^2$, so dividing by $50$ gives
  $159$. $159$ is divisible by $3$, so divide by $3$ to get
  $53$. Since $53$ is prime we are done. Therefore
  \begin{equation*}
    7950 = 2\cdot3\cdot5^2\cdot53. \qedhere
  \end{equation*}
\end{solution}

\section{Problems}

\Problem1 Find the prime-power decompositions of $1234$, $34560$, and
$111111$.
\begin{solution}
  First, $1234$ is divisible by $2$, so we write $1234 =
  2\cdot617$. Now $617$ is not divisible by $2$ or $5$. Using the
  table in Appendix~C, we see that $617$ is prime. Therefore
  $1234 = 2\cdot617$ is the prime factorization.

  For $34560$, first we divide by all factors of $2$ and $5$ to get
  $34560 = 2^8\cdot5\cdot27$. Now $27$ factors as $3^3$ so this gives
  \begin{equation*}
    34560 = 2^8\cdot3^3\cdot5.
  \end{equation*}

  Finally, $111111$ is too big for the table, but by trying small
  possible divisors we can see that it is divisible by $3$, with
  $111111 = 3\cdot37037$. And $37037$ is divisible by $7$:
  $37037 = 7\cdot5291$. Now we may make use of the table to determine
  that $5291$ is divisible by $11$. $5291/11 = 481$, which is
  divisible by $13$. $481/13 = 37$, and $37$ is prime. So
  \begin{equation*}
    111111 = 3\cdot7\cdot11\cdot13\cdot37.\qedhere
  \end{equation*}
\end{solution}

\Problem2 Find the prime-power decompositions of $2345$, $45670$, and
$999999999999$.
\begin{solution}
  Proceeding in the same manner as in the previous problem, we find
  \begin{align*}
    2345 &= 5\cdot7\cdot67, \\
    45670 &= 2\cdot5\cdot4567, \\
    \intertext{and}
    999999999999
         &= 3^3\cdot7\cdot11\cdot13\cdot37\cdot101\cdot9901.
           \qedhere
  \end{align*}
\end{solution}
