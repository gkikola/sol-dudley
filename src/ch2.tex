\chapter{Unique Factorization}

\section{Exercises}

\Exercise1 How many even primes are there? How many whose last digit
is $5$?
\begin{solution}
  If a prime $p$ is even then by definition $2\mid p$. Therefore the
  only prime that is even is $2$ itself. Similarly, any positive
  integer that ends in a $5$ (written in base $10$) must be divisible
  by $5$ (this is due to the fact that our base, $10$, is itself
  divisible by $5$). And the only prime divisible by $5$ is $5$
  itself.
\end{solution}

\Exercise2 Construct a proof of Lemma~$2$ using induction.
\begin{solution}
  Lemma~$2$ says that every positive integer greater than $1$ can be
  written as a product of primes. $2$ is a prime and is a product of
  itself, so the base case is satisfied. Now suppose there is an
  integer $n>1$ such that every integer $k$ with $1<k\leq n$ can be
  written as a product of primes. We must show that $n+1$ can be
  written as such a product.

  If $n+1$ is prime, then we are done, it is already a product of
  primes. If not, then $n+1$ is composite, and we may write $n+1 = st$
  where $s$ and $t$ are each integers with $1<s,t<n+1$. By the
  inductive hypothesis, $s$ and $t$ can each be written as a product
  of primes,
  \begin{equation*}
    s = p_1p_2\cdots p_i,
    \quad\text{and}\quad
    t = q_1q_2\cdots q_j,
  \end{equation*}
  where each $p_k$ and $q_k$ are prime (not necessarily
  distinct). Then
  \begin{equation*}
    n+1 = st = p_1p_2\cdots p_iq_1q_2\cdots q_j,
  \end{equation*}
  and we have written $n+1$ as a product of primes, completing the
  inductive step. It follows by induction that all integers $n>1$ can
  be written as a product of primes.
\end{solution}

\Exercise3 Write prime decompositions for $72$ and $480$.
\begin{solution}
  $72 = 8\cdot9 = 2^3\cdot3^2$ and
  $480 = 48\cdot10 = 16\cdot3\cdot10 = 2^5\cdot3\cdot5$.
\end{solution}

\Exercise4 Which members of the set less than $100$ are not prome?
\begin{solution}
  The set being referenced in the question is the set
  \begin{equation*}
    A = \{4n+1\mid n = 0, 1, 2, \dots\},
  \end{equation*}
  where $k\in A$ is considered ``prome'' if it has no divisors in $A$
  other than $1$ and itself.

  Since $100^{1/2} = 10$, we only need to look for divisors less than
  or equal to $10$. The only such members of $A$ are $1$, $5$, and
  $9$. So any nonprome member of $A$ less than $100$ must be a
  multiple of $5$ or $9$. These numbers are
  \begin{equation*}
    25, 45, 65, 81, 85. \qedhere
  \end{equation*}
\end{solution}

\Exercise5 What is the prime-power decomposition of $7950$?
\begin{solution}
  $7950$ is divisible by $50 = 2\cdot5^2$, so dividing by $50$ gives
  $159$. $159$ is divisible by $3$, so divide by $3$ to get
  $53$. Since $53$ is prime we are done. Therefore
  \begin{equation*}
    7950 = 2\cdot3\cdot5^2\cdot53. \qedhere
  \end{equation*}
\end{solution}

\section{Problems}

\Problem1 Find the prime-power decompositions of $1234$, $34560$, and
$111111$.
\begin{solution}
  First, $1234$ is divisible by $2$, so we write $1234 =
  2\cdot617$. Now $617$ is not divisible by $2$ or $5$. Using the
  table in Appendix~C, we see that $617$ is prime. Therefore
  $1234 = 2\cdot617$ is the prime factorization.

  For $34560$, first we divide by all factors of $2$ and $5$ to get
  $34560 = 2^8\cdot5\cdot27$. Now $27$ factors as $3^3$ so this gives
  \begin{equation*}
    34560 = 2^8\cdot3^3\cdot5.
  \end{equation*}

  Finally, $111111$ is too big for the table, but by trying small
  possible divisors we can see that it is divisible by $3$, with
  $111111 = 3\cdot37037$. And $37037$ is divisible by $7$:
  $37037 = 7\cdot5291$. Now we may make use of the table to determine
  that $5291$ is divisible by $11$. $5291/11 = 481$, which is
  divisible by $13$. $481/13 = 37$, and $37$ is prime. So
  \begin{equation*}
    111111 = 3\cdot7\cdot11\cdot13\cdot37.\qedhere
  \end{equation*}
\end{solution}

\Problem2 Find the prime-power decompositions of $2345$, $45670$, and
$999999999999$.
\begin{solution}
  Proceeding in the same manner as in the previous problem, we find
  \begin{align*}
    2345 &= 5\cdot7\cdot67, \\
    45670 &= 2\cdot5\cdot4567, \\
    \intertext{and}
    999999999999
         &= 3^3\cdot7\cdot11\cdot13\cdot37\cdot101\cdot9901.
           \qedhere
  \end{align*}
\end{solution}

\Problem3 Tartaglia (1556) claimed that the sums
\begin{equation*}
  1 + 2 + 4, \quad 1 + 2 + 4 + 8, \quad 1 + 2 + 4 + 8 + 16,\quad\cdots
\end{equation*}
are alternately prime and composite. Show that he was wrong.
\begin{proof}
  Looking at the partial sums having an odd number of terms, we find
  \begin{align*}
    1 + 2 + 4 &= 7 \\
    1 + 2 + 4 + 8 + 16 &= 31 \\
    1 + 2 + 4 + 8 + 16 + 32 + 64 &= 127 \\
    1 + 2 + 4 + 8 + 16 + 32 + 64 + 128 + 256 &= 511 = 7\cdot73.
  \end{align*}
  Since $511$ is not prime, we see that Tartaglia's conjecture was
  not correct.
\end{proof}

\Problem4
\begin{enumerate}
\item DeBouvelles (1509) claimed that one or both of $6n + 1$ and
  $6n - 1$ are primes for all $n\geq1$. Show that he was wrong.
  \begin{proof}
    For $n = 20$, we have $6n + 1 = 121 = 11^2$ and
    $6n - 1 = 119 = 7\cdot17$. Therefore DeBouvelles's claim is not
    correct.
  \end{proof}
\item Show that there are infinitely many $n$ such that both $6n-1$
  and $6n+1$ are composite.
  \begin{proof}
    Suppose there are finitely many $n$ with both $6n - 1$ and
    $6n + 1$ composite. Let them be $n_1, n_2, \dots, n_k$.

    Now let $n = (6n_k + 9)!$, where $!$ denotes the factorial
    function (i.e., $n! = 1\cdot2\cdot3\cdots(n-1)\cdot n$). Now the
    integers $n+2, n+3, \dots, n+9$ are all composite, since for any
    $m$ with $2\leq m\leq 9$, we clearly have $m \mid n + m$. So we
    have found a sequence of $8$ consecutive composite numbers. Now
    these numbers must include a pair of the form $6t - 1$ and
    $6t + 1$. But both of these are composite, and $t > n_k$. This is
    a contradiction, since $n_k$ was supposed to be the largest such
    value. Therefore there are infinitely many $n$ with both $6n-1$
    and $6n+1$ composite.
  \end{proof}
\end{enumerate}

\Problem5 Prove that if $n$ is a square, then each exponent in its
prime-power decomposition is even.
\begin{proof}
  Let $n>1$ be a square and write $n = k^2$ for some integer
  $k>1$. Let the prime-power decomposition of $k$ be
  \begin{equation*}
    k = p_1^{e_1}p_2^{e_2}\cdots p_r^{e_r}.
  \end{equation*}
  Then
  \begin{align*}
    n &= (p_1^{e_1}p_2^{e_2}\cdots p_r^{e_r})^2 \\
      &= (p_1^{e_1})^2(p_2^{e_2})^2\cdots(p_r^{e_r})^2 \\
      &= p_1^{2e_1}p_2^{2e_2}\cdots p_r^{2e_r}.
  \end{align*}
  Since this prime-power decomposition must be unique (up to
  reordering), we see that every exponent in the prime-power
  decomposition of $n$ is even.
\end{proof}

\Problem6 Prove that if each exponent in the prime-power decomposition
of $n$ is even, then $n$ is a square.
\begin{proof}
  Suppose every exponent in the prime-power decomposition of $n$ is
  even. Then each exponent $e_i$ in the decomposition has the form
  $e_i = 2f_i$ for some integer $f_i$. Then $n$ can be written
  \begin{align*}
    n &= p_1^{2f_1}p_2^{2f_2}\cdots p_r^{2f_r} \\
      &= (p_1^{f_1})^2(p_2^{f_2})^2\cdots(p_r^{f_r})^2 \\
      &= (p_1^{f_1}p_2^{f_2}\cdots p_r^{f_r})^2 \\
      &= k^2,
  \end{align*}
  where $k = p_1^{f_1}\cdots p_r^{f_r}$, and we see that $n$ is a
  square.
\end{proof}

\Problem7 Find the smallest integer divisible by $2$ and $3$ which is
simultaneously a square and a fifth power.
\begin{solution}
  Let the smallest such number be $n$. The least common multiple of
  $2$ and $3$ is $6$, so $6\mid n$. $n$ is a square and a fifth power,
  so $n$ must actually be a tenth power, since $10$ is the least
  common multiple of $2$ and $5$. The smallest tenth power divisible
  by $6$ is $6^{10}$, so we have
  \begin{equation*}
    n = 6^{10} = 60466176. \qedhere
  \end{equation*}
\end{solution}

\Problem8 If $d\mid ab$, does it follow that $d\mid a$ or $d\mid b$?
\begin{solution}
  No. For example, $6\mid 4\cdot9$ but $6\nmid4$ and $6\nmid9$. If,
  however, we know that $d$ is prime, then the conclusion {\em does}
  hold, as proved in Lemma~5.
\end{solution}

\Problem9 Is it possible for a prime $p$ to divide both $n$ and $n+1$
($n\geq1$)?
\begin{solution}
  No. For, if it is possible, suppose the prime $p$ divides both $n$
  and $n+1$. Then $p$ also divides their difference, $(n+1) - n =
  1$. So we would have $p\mid1$, which is clearly absurd.
\end{solution}

\Problem{10} Prove that $n(n+1)$ is never a square for $n>0$.
\begin{proof}
  Suppose $n(n+1) = k^2$ for some integer $k>0$. Then $n^2+n = k^2$
  which gives $k^2 - n^2 = n$. Factoring the left-hand side then gives
  \begin{equation*}
    (k+n)(k-n) = n.
  \end{equation*}
  So in particular, $k+n\mid n$. But this is impossible, since
  $k+n > n > 0$. This contradiction shows that $n(n+1)$ is not a
  square.
\end{proof}
