\chapter{Unique Factorization}

\section{Exercises}

\Exercise1 How many even primes are there? How many whose last digit
is $5$?
\begin{solution}
  If a prime $p$ is even then by definition $2\mid p$. Therefore the
  only prime that is even is $2$ itself. Similarly, any positive
  integer that ends in a $5$ (written in base $10$) must be divisible
  by $5$ (this is due to the fact that our base, $10$, is itself
  divisible by $5$). And the only prime divisible by $5$ is $5$
  itself.
\end{solution}

\Exercise2 Construct a proof of Lemma~$2$ using induction.
\begin{solution}
  Lemma~$2$ says that every positive integer greater than $1$ can be
  written as a product of primes. $2$ is a prime and is a product of
  itself, so the base case is satisfied. Now suppose there is an
  integer $n>1$ such that every integer $k$ with $1<k\leq n$ can be
  written as a product of primes. We must show that $n+1$ can be
  written as such a product.

  If $n+1$ is prime, then we are done, it is already a product of
  primes. If not, then $n+1$ is composite, and we may write $n+1 = st$
  where $s$ and $t$ are each integers with $1<s,t<n+1$. By the
  inductive hypothesis, $s$ and $t$ can each be written as a product
  of primes,
  \begin{equation*}
    s = p_1p_2\cdots p_i,
    \quad\text{and}\quad
    t = q_1q_2\cdots q_j,
  \end{equation*}
  where each $p_k$ and $q_k$ are prime (not necessarily
  distinct). Then
  \begin{equation*}
    n+1 = st = p_1p_2\cdots p_iq_1q_2\cdots q_j,
  \end{equation*}
  and we have written $n+1$ as a product of primes, completing the
  inductive step. It follows by induction that all integers $n>1$ can
  be written as a product of primes.
\end{solution}

\Exercise3 Write prime decompositions for $72$ and $480$.
\begin{solution}
  $72 = 8\cdot9 = 2^3\cdot3^2$ and
  $480 = 48\cdot10 = 16\cdot3\cdot10 = 2^5\cdot3\cdot5$.
\end{solution}
