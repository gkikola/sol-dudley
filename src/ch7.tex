\chapter{The Divisors of an Integer}

\section{Exercises}

\Exercise1 Verify that the table of values for $d(n)$ is correct as
far as it goes, and complete it.
\begin{solution}
  For small values of $n$, we can simply test divisibility by each
  positive integer up to $n/2$. The completed table follows.
  \begin{center}
    \begin{tabular}{r|rrrrrrrrrrrrrrrr}
      $n$ & 1 & 2 & 3 & 4 & 5 & 6 & 7 & 8 & 9 & 10
      & 11 & 12 & 13 & 14 & 15 & 16 \\
      $d(n)$ & 1 & 2 & 2 & 3 & 2 & 4 & 2 & 4 & 3 & 4
      & 2 & 6 & 2 & 4 & 4 & 5
    \end{tabular}
  \end{center}
\end{solution}

\Exercise2 What is $d(p^3)$? Generalize to $d(p^n)$,
$n = 4, 5, \dots$.
\begin{solution}
  For any prime $p$, the divisors of $p^3$ are $1$, $p$, $p^2$, and
  $p^3$. These are the only divisors, so $d(p^3) = 4$.

  In general, $d(p^n) = n + 1$. For, if $p^n$ has a divisor $a$ that
  is not of the form $p^i$ for some nonnegative $i$, then this divisor
  $a$ must contain some prime factor $q$ distinct from $p$. But if
  $q\mid a$ then $q\mid p^n$. By Lemma~6 of Section~2, it follows that
  $q\mid p$ as well. Therefore $p$ is a prime which contains a smaller
  prime as a factor, which is impossible. This contradiction shows
  that the only divisors of $p^n$ have the form $p^i$, where
  $i = 0, 1, \dots, n$.
\end{solution}

\Exercise3 What is $d(p^3q)$? What is $d(p^nq)$ for any positive $n$?
\begin{solution}
  $p^3q$ has as its divisors $1$, $p$, $p^2$, and $p^3$, and also $q$,
  $pq$, $p^2q$, and $p^3q$, for a total of $8$ divisors.

  In general, every divisor of $p^n$ will be a divisor of $p^nq$, and
  for each such divisor, multiplying by $q$ will produce a new
  divisor. So the factor of $q$ effectively doubles the number of
  divisors. Consequently, we have
  \begin{equation*}
    d(p^nq) = 2(n + 1) = 2n + 2. \qedhere
  \end{equation*}
\end{solution}
