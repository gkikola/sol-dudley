\chapter{Congruences}

\section{Exercises}

\Exercise1 True or false? $91\equiv0\pmod7$.
$3 + 5 + 7 \equiv 5\pmod{10}$. $-2\equiv2\pmod8$. $11^2\equiv1\pmod3$.
\begin{solution}
  Only the third congruence is false.

  Since $91 = 7\cdot13$ we have $7\mid(91-0)$ so that
  $91\equiv0\pmod7$.

  $3 + 5 + 7 = 15$ and $10\mid(15-5)$ so $3 + 5 + 7 \equiv 5\pmod{10}$.

  It is not true that $-2\equiv2\pmod8$, since $8\nmid-4$.

  And since $3\mid(121 - 1)$, we indeed have $11^2\equiv1\pmod3$.
\end{solution}

\Exercise2 Complete the proof that $a\equiv b\pmod m$ if and only if
there is an integer $k$ such that $a = b + km$.
\begin{solution}
  In the text, Dudley proves the left-to-right implication. So we need
  to show the converse. Suppose that $a = b + km$ for some integer
  $k$. Then $a - b = km$ and we have by the definition of divisibility
  that $m\mid(a - b)$. Therefore $a\equiv b\pmod{m}$.
\end{solution}

\Exercise3 To what least residue (mod $11$) is each of $23$, $29$,
$31$, $37$, and $41$ congruent?
\begin{solution}
  We have
  \begin{align*}
    23 &\equiv 1\pmod{11}, \\
    29 &\equiv 7\pmod{11}, \\
    31 &\equiv 9\pmod{11}, \\
    37 &\equiv 4\pmod{11}, \\
    \intertext{and}
    41 &\equiv 8\pmod{11}. \qedhere
  \end{align*}
\end{solution}

\Exercise4 Say ``$n$ is odd'' in three other ways.
\begin{solution}
  From the theorems in the text, $n$ is odd if and only
  $n\equiv1\pmod2$, if and only if $n = 2k + 1$ for some integer $k$,
  if and only if $n$ has remainder $1$ when divided by $2$.
\end{solution}

\Exercise5 Prove that $p\mid a$ if and only if $a\equiv0\pmod{p}$.
\begin{proof}
  This is immediate from the definition of congruence, since $p\mid a$
  if and only if $p\mid(a-0)$.
\end{proof}

\Exercise6 Prove that $a\equiv a\pmod{m}$ for all integers $a$.
\begin{proof}
  Since any positive integer $m$ must divide $0$, we have
  $m\mid(a - a)$ so that $a\equiv a\pmod{m}$.
\end{proof}

\Exercise7 Prove that for all integers $a$ and $b$, if
$a\equiv b\pmod{m}$, then $b\equiv a\pmod{m}$.
\begin{proof}
  If $a\equiv b\pmod{m}$ then $a - b = km$ for some integer $k$. Then
  we also have $b - a = (-k)m$ so that $b\equiv a\pmod{m}$.
\end{proof}

\Exercise8 Prove that for integers $a$, $b$, and $c$, if
$a\equiv b\pmod m$ and $b\equiv c\pmod m$, then $a\equiv c\pmod m$.
\begin{proof}
  By definition, there are integers $s$ and $t$ with $a - b = sm$ and
  $b - c = tm$. So
  \begin{equation*}
    a - c = (a - b) + (b - c) = sm + tm = (s + t)m,
  \end{equation*}
  hence $m\mid(a - c)$.
\end{proof}

\Exercise9 Prove that for integers $a$, $b$, $c$, and $d$, if
$a\equiv b\pmod m$ and $c\equiv d\pmod m$, then
$a + c\equiv b + d\pmod m$.
\begin{proof}
  We have $a = b + sm$ and $c = d + tm$ for integers $s$ and $t$. So
  \begin{equation*}
    a + c = (b + sm) + (d + tm) = (b + d) + (s + t)m,
  \end{equation*}
  and we see that $a + c\equiv b + d\pmod m$.
\end{proof}

\Exercise{10} Construct a like example for modulus $10$ to show that
$ab\equiv ac\pmod m$ and $a\not\equiv0\pmod m$ do not together imply
$b\equiv c\pmod m$.
\begin{solution}
  We have $5\cdot2\equiv5\cdot4\pmod{10}$ and $5\not\equiv0\pmod{10}$,
  but $2\not\equiv4\pmod{10}$.
\end{solution}

\Exercise{11} What values of $x$ satisfy
\begin{enumerate}
\item $2x\equiv4\pmod7$?
  \begin{solution}
    Since $(2,7) = 1$, we are allowed (by Theorem~4) to cancel a
    factor of $2$ on each side to get $x\equiv2\pmod7$.
  \end{solution}
\item $2x\equiv1\pmod7$?
  \begin{solution}
    Since $1\equiv8\pmod7$, we can again cancel a factor of $2$ to get
    $x\equiv4\pmod7$.
  \end{solution}
\end{enumerate}

\Exercise{12} Which $x$ will satisfy $2x\equiv4\pmod6$?
\begin{solution}
  We have $(2,6) = 2$. Applying Theorem~5, we get
  \begin{equation*}
    x\equiv2\pmod3. \qedhere
  \end{equation*}
\end{solution}

\section{Problems}

\Problem1 Find the least residue of $1492$ (mod $4$), (mod $10$), and
(mod $101$).
\begin{solution}
  Since $1492 = 4\cdot373$ we have $1492\equiv0\pmod4$. Since its last
  decimal digit is $2$, we know that $1492\equiv2\pmod{10}$. Finally,
  since $1492 = 14\cdot101 + 78$, we have $1492\equiv78\pmod{101}$.
\end{solution}

\Problem2 Find the least residue of $1789$ (mod $4$), (mod $10$), and
(mod $101$).
\begin{solution}
  We have $1789\equiv1\pmod4$, $1789\equiv9\pmod{10}$, and
  $1789\equiv72\pmod{101}$.
\end{solution}

\Problem3 Prove or disprove that if $a\equiv b\pmod m$, then
$a^2\equiv b^2\pmod m$.
\begin{solution}
  This is true. The proof is immediate from part (e) of Lemma~1.
\end{solution}

\Problem4 Prove or disprove that if $a^2\equiv b^2\pmod m$, then
$a\equiv b$ or $-b\pmod m$.
\begin{solution}
  This is not true in general. For a counterexample, take $m = 12$. We
  have $2^2\equiv4^2\pmod{12}$ but $2\not\equiv4\pmod{12}$ and
  $2\not\equiv-4\equiv8\pmod{12}$.
\end{solution}

\Problem5 Find all $m$ such that $1066\equiv1776\pmod{m}$.
\begin{solution}
  We need $m$ to divide $1776 - 1066 = 710$. Since
  $710 = 2\cdot5\cdot71$, the possible values of $m$ are $1$, $2$,
  $5$, $10$, $71$, $142$, $355$, and $710$.
\end{solution}

\Problem6 Find all $m$ such that $1848\equiv1914\pmod{m}$.
\begin{solution}
  $1914 - 1848 = 66$ which factors as $66 = 2\cdot3\cdot11$, so the
  possible values for $m$ are $1$, $2$, $3$, $6$, $11$, $22$, $33$, or
  $66$.
\end{solution}

\Problem7 If $k\equiv1\pmod4$, then what is $6k+5$ congruent to (mod
$4$)?
\begin{solution}
  From Lemma~1, we have
  \begin{equation*}
    6k+5\equiv6\cdot1+5\equiv11\equiv3\pmod4. \qedhere
  \end{equation*}
\end{solution}

\Problem8 Show that every prime (except $2$) is congruent to $1$ or
$3$ (mod $4$).
\begin{solution}
  Let $p$ be a prime bigger than $2$. Since no prime (other than $2$)
  is divisible by $2$, we cannot have $p = 4k$ or $p = 4k + 2$ for an
  integer $k$. So $4\nmid p$ and $4\nmid(p-2)$. Therefore
  $p\not\equiv0\pmod4$ and $p\not\equiv2\pmod4$. The only remaining
  possibilities are $p\equiv1$ or $3\pmod4$.
\end{solution}

\Problem9 Show that every prime (except $2$ or $3$) is congruent to
$1$ or $5$ (mod $6$).
\begin{solution}
  Let $p$ be a prime larger than $3$. If $p\equiv0\pmod6$, then
  $6\mid p$ which is impossible. If $p\equiv2\pmod6$ then
  $p = 6k + 2 = 2(3k + 1)$ for an integer $k$, which is impossible. If
  $p\equiv3\pmod6$ then $p = 6k + 3 = 3(2k + 1)$, which is
  impossible. And if $p\equiv4\pmod6$ then $p = 6k + 4 = 2(3k + 2)$,
  which is again impossible. The only possibilities are $p\equiv1$ or
  $5\pmod6$.
\end{solution}

\Problem{10} What can primes (except $2$, $3$, or $5$) be congruent to
(mod $30$)?
\begin{solution}
  Let $p$ be a prime greater than $5$ and let $k$ be a nonnegative
  integer less than $30$. If $p\equiv k\pmod{30}$ then $p = 30n + k$
  for some integer $n$. From this we see that $p$ cannot be prime
  (larger than $5$) unless $(30,k) = 1$ (since the factors of $30$ are
  $2$, $3$, and $5$, and primes larger than $5$ cannot be divisible by
  these numbers). So the possible values for $k$ are those that are
  relatively prime to $30$, namely $1$, $7$, $11$, $13$, $17$, $19$,
  $23$, or $29$.
\end{solution}

\Problem{11} In the multiplication $31415\cdot92653 = 2910\;93995$,
one digit in the product is missing and all the others are
correct. Find the missing digit without doing the multiplication.
\begin{solution}
  By repeatedly summing the digits, we see that
  \begin{equation*}
    31415\cdot92653\equiv14\cdot25\equiv5\cdot7\equiv35\equiv8\pmod9.
  \end{equation*}
  Using $k$ in place of the missing digit in the product, we have
  \begin{equation*}
    2910k93995 \equiv 47+k\equiv 11+k\equiv2+k\pmod9.
  \end{equation*}
  So $2+k\equiv8\pmod9$ and we see that $k$ must be $6$.
\end{solution}

\Problem{12} Show that no square has as its last digit, $2$, $3$, $7$,
or $8$.
\begin{proof}
  Let $n$ be any nonnegative integer. Modulo $10$, there are only $10$
  possible least residues for $n$, so we may simply square each of
  them and reduce:
  \begin{align*}
    n\equiv0\pmod{10}\quad&\Rightarrow\quad n^2\equiv0\pmod{10}, \\
    n\equiv1\pmod{10}\quad&\Rightarrow\quad n^2\equiv1\pmod{10}, \\
    n\equiv2\pmod{10}\quad&\Rightarrow\quad n^2\equiv4\pmod{10}, \\
    n\equiv3\pmod{10}\quad&\Rightarrow\quad n^2\equiv9\pmod{10}, \\
    n\equiv4\pmod{10}\quad&\Rightarrow\quad n^2\equiv6\pmod{10}, \\
    n\equiv5\pmod{10}\quad&\Rightarrow\quad n^2\equiv5\pmod{10}, \\
    n\equiv6\pmod{10}\quad&\Rightarrow\quad n^2\equiv6\pmod{10}, \\
    n\equiv7\pmod{10}\quad&\Rightarrow\quad n^2\equiv9\pmod{10}, \\
    n\equiv8\pmod{10}\quad&\Rightarrow\quad n^2\equiv4\pmod{10}, \\
    n\equiv9\pmod{10}\quad&\Rightarrow\quad n^2\equiv1\pmod{10}.
  \end{align*}
  We see in each case that $n^2$ can only have $0$, $1$, $4$, $5$,
  $6$, or $9$ as its last digit.
\end{proof}

\Problem{13} What can the last digit of a fourth power be?
\begin{solution}
  We simply raise each least residue (mod $10$) to the fourth power,
  similar to what we did in the previous problem. Modulo $10$, we have
  $0^4 = 0$, $1^4 = 1$, $2^4 = 16\equiv6$, $3^4 = 81\equiv1$, and so
  on. After going through all the digits, we can see that the only
  possibilities for the last digit of a fourth power are $0$, $1$,
  $5$, or $6$.
\end{solution}

\Problem{14} Show that the difference of two consecutive cubes is
never divisible by $3$.
\begin{proof}
  Let $n$ be an integer. We have
  \begin{align*}
    (n+1)^3 - n^3
    &= n^3 + 3n^2 + 3n + 1 - n^3 \\
    &= 3n^2 + 3n + 1 \\
    &\equiv1\pmod3.
  \end{align*}
  Since $(n+1)^3 - n^3$ always has a remainder of $1$ when divided by
  $3$, it cannot be divisible by $3$.
\end{proof}

\Problem{15} Show that the difference of two consecutive cubes is
never divisible by $5$.
\begin{proof}
  Let $n$ be an integer. As in the previous problem,
  \begin{equation*}
    (n+1)^3 - n^3 = 3n^2 + 3n + 1.
  \end{equation*}
  We find that
  \begin{align*}
    3(0)^2 + 3(0) + 1 = 1 &\equiv1\pmod5, \\
    3(1)^2 + 3(1) + 1 = 7 &\equiv2\pmod5, \\
    3(2)^2 + 3(2) + 1 = 19 &\equiv4\pmod5, \\
    3(3)^2 + 3(3) + 1 = 37 &\equiv2\pmod5, \\
    \intertext{and}
    3(4)^2 + 3(4) + 1 = 61 &\equiv1\pmod5.
  \end{align*}
  So, if $n$ is congruent (mod $5$) to $0$, $1$, $2$, $3$, or $4$,
  then $(n+1)^3-n^3$ is not divisible by $5$. But $n$ must be
  congruent to one of these, so we have checked every case.
\end{proof}

\Problem{16} Show that
\begin{multline}
  \label{eq:congruences:digits-mod-11}
  d_k10^k + d_{k-1}10^{k-1} + \cdots + d_110 + d_0 \\
  \equiv d_0 - d_1 + d_2 - d_3 + \cdots + (-1)^kd_k\pmod{11}
\end{multline}
and deduce a test for divisibility by $11$.
\begin{solution}
  Since $10\equiv-1\pmod{11}$, it follows that $10^n\equiv1\pmod{11}$
  when $n$ is even and $10^n\equiv-1\pmod{11}$ when $n$ is odd. So any
  positive integer is congruent (mod $11$) to the sum of its digits
  but with alternating signs. Therefore
  \eqref{eq:congruences:digits-mod-11} holds.

  To test for divisibility by $11$, simply find the sum of every other
  digit, and subtract the sum of the remaining digits. Then this
  difference is divisible by $11$ if and only if the original number
  is as well.

  For example, 37,536,760,679 is divisible by $11$ since
  \begin{equation*}
    3 + 5 + 6 + 6 + 6 + 9 = 35,
  \end{equation*}
  \begin{equation*}
    7 + 3 + 7 + 0 + 7 = 24,
  \end{equation*}
  and $35 - 24 = 11$ is divisible by $11$.
\end{solution}

\Problem{17} $A$ says, ``27,182,818,284,590,452 is divisible by
$11$.'' $B$ says, ``No, it isn't.'' Who is right?
\begin{solution}
  We may simply use the divisibility rule found in the previous
  problem. The cross-digit sums are
  \begin{equation*}
    2 + 1 + 2 + 1 + 2 + 4 + 9 + 4 + 2 = 27
  \end{equation*}
  and
  \begin{equation*}
    7 + 8 + 8 + 8 + 8 + 5 + 0 + 5 = 49.
  \end{equation*}
  Since $49 - 27 = 22$ and $11\mid22$, we see that the original number
  is divisible by $11$. Therefore $A$'s assertion is correct.
\end{solution}
