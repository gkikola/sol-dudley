\chapter{Congruences}

\section{Exercises}

\Exercise1 True or false? $91\equiv0\pmod7$.
$3 + 5 + 7 \equiv 5\pmod{10}$. $-2\equiv2\pmod8$. $11^2\equiv1\pmod3$.
\begin{solution}
  Only the third congruence is false.

  Since $91 = 7\cdot13$ we have $7\mid(91-0)$ so that
  $91\equiv0\pmod7$.

  $3 + 5 + 7 = 15$ and $10\mid(15-5)$ so $3 + 5 + 7 \equiv 5\pmod{10}$.

  It is not true that $-2\equiv2\pmod8$, since $8\nmid-4$.

  And since $3\mid(121 - 1)$, we indeed have $11^2\equiv1\pmod3$.
\end{solution}

\Exercise2 Complete the proof that $a\equiv b\pmod m$ if and only if
there is an integer $k$ such that $a = b + km$.
\begin{solution}
  In the text, Dudley proves the left-to-right implication. So we need
  to show the converse. Suppose that $a = b + km$ for some integer
  $k$. Then $a - b = km$ and we have by the definition of divisibility
  that $m\mid(a - b)$. Therefore $a\equiv b\pmod{m}$.
\end{solution}

\Exercise3 To what least residue (mod $11$) is each of $23$, $29$,
$31$, $37$, and $41$ congruent?
\begin{solution}
  We have
  \begin{align*}
    23 &\equiv 1\pmod{11}, \\
    29 &\equiv 7\pmod{11}, \\
    31 &\equiv 9\pmod{11}, \\
    37 &\equiv 4\pmod{11}, \\
    \intertext{and}
    41 &\equiv 8\pmod{11}. \qedhere
  \end{align*}
\end{solution}

\Exercise4 Say ``$n$ is odd'' in three other ways.
\begin{solution}
  From the theorems in the text, $n$ is odd if and only
  $n\equiv1\pmod2$, if and only if $n = 2k + 1$ for some integer $k$,
  if and only if $n$ has remainder $1$ when divided by $2$.
\end{solution}

\Exercise5 Prove that $p\mid a$ if and only if $a\equiv0\pmod{p}$.
\begin{proof}
  This is immediate from the definition of congruence, since $p\mid a$
  if and only if $p\mid(a-0)$.
\end{proof}

\Exercise6 Prove that $a\equiv a\pmod{m}$ for all integers $a$.
\begin{proof}
  Since any positive integer $m$ must divide $0$, we have
  $m\mid(a - a)$ so that $a\equiv a\pmod{m}$.
\end{proof}

\Exercise7 Prove that for all integers $a$ and $b$, if
$a\equiv b\pmod{m}$, then $b\equiv a\pmod{m}$.
\begin{proof}
  If $a\equiv b\pmod{m}$ then $a - b = km$ for some integer $k$. Then
  we also have $b - a = (-k)m$ so that $b\equiv a\pmod{m}$.
\end{proof}

\Exercise8 Prove that for integers $a$, $b$, and $c$, if
$a\equiv b\pmod m$ and $b\equiv c\pmod m$, then $a\equiv c\pmod m$.
\begin{proof}
  By definition, there are integers $s$ and $t$ with $a - b = sm$ and
  $b - c = tm$. So
  \begin{equation*}
    a - c = (a - b) + (b - c) = sm + tm = (s + t)m,
  \end{equation*}
  hence $m\mid(a - c)$.
\end{proof}

\Exercise9 Prove that for integers $a$, $b$, $c$, and $d$, if
$a\equiv b\pmod m$ and $c\equiv d\pmod m$, then
$a + c\equiv b + d\pmod m$.
\begin{proof}
  We have $a = b + sm$ and $c = d + tm$ for integers $s$ and $t$. So
  \begin{equation*}
    a + c = (b + sm) + (d + tm) = (b + d) + (s + t)m,
  \end{equation*}
  and we see that $a + c\equiv b + d\pmod m$.
\end{proof}

\Exercise{10} Construct a like example for modulus $10$ to show that
$ab\equiv ac\pmod m$ and $a\not\equiv0\pmod m$ do not together imply
$b\equiv c\pmod m$.
\begin{solution}
  We have $5\cdot2\equiv5\cdot4\pmod{10}$ and $5\not\equiv0\pmod{10}$,
  but $2\not\equiv4\pmod{10}$.
\end{solution}

\Exercise{11} What values of $x$ satisfy
\begin{enumerate}
\item $2x\equiv4\pmod7$?
  \begin{solution}
    Since $(2,7) = 1$, we are allowed (by Theorem~4) to cancel a
    factor of $2$ on each side to get $x\equiv2\pmod7$.
  \end{solution}
\item $2x\equiv1\pmod7$?
  \begin{solution}
    Since $1\equiv8\pmod7$, we can again cancel a factor of $2$ to get
    $x\equiv4\pmod7$.
  \end{solution}
\end{enumerate}

\Exercise{12} Which $x$ will satisfy $2x\equiv4\pmod6$?
\begin{solution}
  We have $(2,6) = 2$. Applying Theorem~5, we get
  \begin{equation*}
    x\equiv2\pmod3. \qedhere
  \end{equation*}
\end{solution}
