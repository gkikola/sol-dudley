\chapter{Linear Congruences}

\section{Exercises}

\Exercise1 Construct congruences modulo $12$ with no solutions, just
one solution, and more than one solution.
\begin{solution}
  The congruence $3x\equiv4\pmod{12}$ has no solution since $3x$ is
  always congruent to $0$, $3$, $6$, or $9$ (mod $12$) and never
  $4$. The congruence $5x\equiv2\pmod{12}$ has one solution, $x =
  10$. The congruence $2x\equiv4\pmod{12}$ has two solutions, $x = 2$
  and $x = 8$.
\end{solution}

\Exercise2 Which congruences have no solutions?
\label{exercise:lin-cong:which-congruences-have-solutions}
\begin{enumerate}
\item $3x\equiv1\pmod{10}$,
\item $4x\equiv1\pmod{10}$,
\item $5x\equiv1\pmod{10}$,
\item $6x\equiv1\pmod{10}$,
\item $7x\equiv1\pmod{10}$.
\end{enumerate}
\begin{solution}
  Since $3\cdot7 = 21\equiv1\pmod{10}$, both of the congruences
  $3x\equiv1$ and $7x\equiv1\pmod{10}$ have a solution.

  The other three congruences do not have solutions.
\end{solution}

\Exercise3 After
Exercise~\ref{exercise:lin-cong:which-congruences-have-solutions}, can
you guess a criterion for telling when a congruence has no solutions?
\begin{solution}
  A necessary and sufficient condition that a congruence
  $ax\equiv b\pmod{m}$ has no solutions is that $(a,m)$ does not
  divide $b$. This will be proven in Lemma~1 in the text.
\end{solution}

\Exercise4 Solve
\begin{enumerate}
\item $8x\equiv1\pmod{15}$
  \begin{solution}
    Since $(8,15) = 1$, there is only one solution (by Lemma~2). We
    have $8x\equiv16\pmod{15}$ so that $x\equiv2\pmod{15}$.
  \end{solution}
\item $9x + 10y = 11$
  \begin{solution}
    From the equation we get the congruence
    $9x\equiv11\pmod{10}$. Since $11\equiv81\pmod{10}$, we have
    $9x\equiv81\pmod{10}$ from which we get $x\equiv9\pmod{10}$. This
    is the only solution to the congruence. Thus
    \begin{equation*}
      x = 9 + 10t
    \end{equation*}
    gives all possible values for $x$. Substituting this back into the
    equation, we get
    \begin{equation*}
      9(9 + 10t) + 10y = 11,
    \end{equation*}
    which gives
    \begin{equation*}
      y = -7 - 9t.
    \end{equation*}
    So the general solution is $x = 9 + 10t$ and $y = -7 - 9t$.
  \end{solution}
\end{enumerate}

\Exercise5
\label{exercise:lin-cong:sampler}
Determine the number of solutions of each of the following
congruences:
\begin{gather*}
  3x\equiv6\pmod{15},\qquad 4x\equiv8\pmod{15},\qquad 5x\equiv10\pmod{15}, \\
  6x\equiv11\pmod{15},\qquad 7x\equiv14\pmod{15}.
\end{gather*}
\begin{solution}
  We will use Lemma~3.

  $(3,15) = 3$ and $3\mid6$, so $3x\equiv6\pmod{15}$ has $3$
  solutions.

  $(4,15) = 1$, so $4x\equiv8\pmod{15}$ has only one solution.

  $(5,15) = 5$ and $5\mid{10}$, so $5x\equiv10\pmod{15}$ has $5$
  solutions.

  $(6,15) = 3$ but $3\nmid11$ so the congruence $6x\equiv11\pmod{15}$
  has no solutions.

  Finally, since $(7,15) = 1$, the congruence $7x\equiv14\pmod{15}$
  has one solution.
\end{solution}

\Exercise6 Find all of the solutions of $5x\equiv10\pmod{15}$.
\begin{solution}
  Since $(5,10) = 5$, we may divide by $5$ to get $x\equiv2\pmod3$. So
  the solutions modulo $15$ are $2$, $5$, $8$, $11$, and $14$.
\end{solution}

\Exercise7 Solve the rest of the congruences in
Exercise~\ref{exercise:lin-cong:sampler}.
\begin{solution}
  From $3x\equiv6\pmod{15}$ we get $x\equiv2\pmod{5}$, so that the
  three solutions modulo $15$ are $2$, $7$, and $12$.

  For $4x\equiv8\pmod{15}$ we get the unique solution $x = 2$.

  As we saw before, $6x\equiv11\pmod{15}$ has no solutions since
  $(6,15)\nmid11$.

  Lastly, for $7x\equiv14\pmod{15}$ we have the unique solution
  $x = 2$.
\end{solution}

\Exercise8 Verify that $52$ satisfies each of the three congruences,
$x\equiv1\pmod3$, $x\equiv2\pmod5$, and $x\equiv3\pmod7$.
\begin{solution}
  Since
  \begin{equation*}
    52 = 17\cdot3 + 1 = 10\cdot5 + 2 = 7\cdot7 + 3,
  \end{equation*}
  we see that each congruence is satisfied.
\end{solution}

\section{Problems}

\Problem1 Solve each of the following:
\begin{gather*}
  2x\equiv1\pmod{17}. \qquad\qquad 3x\equiv1\pmod{17}. \\
  3x\equiv6\pmod{18}. \qquad\qquad 40x\equiv777\pmod{1777}.
\end{gather*}
\begin{solution}
  For the first congruence, we have $2x\equiv1\equiv18\pmod{17}$ so
  $x\equiv9\pmod{17}$. This is the only solution, since $(2,17) = 1$.

  For the second, we have $3x\equiv1\equiv18\pmod{17}$ so
  $x\equiv6\pmod{17}$ and again, this solution is unique.

  For the third congruence, we may divide by the greatest common
  divisor to get $x\equiv2\pmod6$. So the $3$ solutions modulo $18$
  are $2$, $8$, and $14$.

  Lastly, $(40,1777) = 1$ so we do have a unique solution. Since
  \begin{equation*}
    40x\equiv777\equiv-1000\pmod{1777},
  \end{equation*}
  we may divide by $40$ to get
  \begin{equation*}
    x\equiv-25\pmod{1777}.
  \end{equation*}
  Therefore $x = 1752$ is the only least residue satisfying the
  congruence.
\end{solution}

\Problem2 Solve each of the following:
\begin{gather*}
  2x\equiv1\pmod{19}. \qquad\qquad 3x\equiv1\pmod{19}. \\
  4x\equiv6\pmod{18}. \qquad\qquad 20x\equiv984\pmod{1984}.
\end{gather*}
\begin{solution}
  For the first congruence, we have $2x\equiv1\equiv20\pmod{19}$ so
  that $x\equiv10\pmod{19}$, and this is the only solution.

  For the second, we have $3x\equiv1\equiv39\pmod{19}$ so
  $x\equiv13\pmod{19}$, and this is again the only solution.

  For the third, we get $2x\equiv3\equiv12\pmod9$ so that
  $x\equiv6\pmod9$. The two solutions (mod $18$) are then $6$ and
  $15$.

  Finally, for the last congruence, we may divide by $4$ to get
  $5x\equiv246\pmod{496}$. So $5x\equiv-250\pmod{496}$ and we get
  $x\equiv-50\pmod{496}$. The four solutions modulo $1984$ are then
  $x = 446$, $942$, $1438$, and $1934$.
\end{solution}

\Problem3 Solve the systems
\begin{enumerate}
\item $x\equiv1\pmod2$, $x\equiv1\pmod3$.
  \begin{solution}
    If $x\equiv1\pmod2$, then
    \begin{equation*}
      x = 1 + 2k_1
      \quad\text{for some integer $k_1$}.
    \end{equation*}
    So if $x\equiv1\pmod3$ then $1 + 2k_1\equiv1\pmod3$ or
    $k_1\equiv0\pmod3$. So $k_1 = 3k_2$ for some $k_2$. Therefore
    \begin{equation*}
      x = 1 + 6k_2,
      \quad\text{or}\quad
      x\equiv1\pmod6.
    \end{equation*}
    By the Chinese Remainder Theorem, this is the only solution modulo
    $6$.
  \end{solution}
\item $x\equiv3\pmod5$, $x\equiv5\pmod7$, $x\equiv7\pmod{11}$.
  \begin{solution}
    From the first congruence we get
    \begin{equation*}
      x = 3 + 5k_1.
    \end{equation*}
    So by the second congruence we have $3 + 5k_1\equiv5\pmod7$ and
    solving this gives $k_1\equiv6\pmod7$, so that
    \begin{equation*}
      x = 3 + 5(6 + 7k_2) = 33 + 35k_2.
    \end{equation*}
    Finally, using the last congruence we get
    $33 + 35k_2\equiv7\pmod{11}$ or $2k_2\equiv7\pmod{11}$. Solving
    this gives $k_2\equiv9\pmod{11}$, so we get
    \begin{equation*}
      x = 33 + 35(9 + 11k_3) = 348 + 385k_3.
    \end{equation*}
    So, $x\equiv348\pmod{385}$ and this solution is unique modulo
    $385$.
  \end{solution}
\item $2x\equiv1\pmod5$, $3x\equiv2\pmod7$, $4x\equiv3\pmod{11}$.
  \begin{solution}
    The first congruence gives $x = 3 + 5k_1$ for some
    $k_1$. Substituting this into the second congruence gives
    $k_1 = 7k_2$, so that $x = 3 + 35k_2$. Using the third
    congruence, we get $k_2 = 3 + 11k_3$. So
    \begin{equation*}
      x = 108 + 385k_3.
    \end{equation*}
    Therefore $x\equiv108\pmod{385}$.
  \end{solution}
\end{enumerate}

\Problem4 Solve the systems
\begin{enumerate}
\item $x\equiv1\pmod2$, $x\equiv2\pmod3$.
  \begin{solution}
    By inspection, we see that $x\equiv5\pmod6$ is a solution, and by
    the Chinese Remainder Theorem this is the only solution.
  \end{solution}
\item $x\equiv2\pmod5$, $2x\equiv3\pmod7$, $3x\equiv4\pmod{11}$.
  \begin{solution}
    Using the same technique as in the previous problem, we find that
    $x\equiv82\pmod{385}$.
  \end{solution}
\item $x\equiv31\pmod{41}$, $x\equiv59\pmod{26}$.
  \begin{solution}
    Again, using the same familiar method as before we get
    $x\equiv605\pmod{1066}$.
  \end{solution}
\end{enumerate}

\Problem5 What possibilities are there for the number of solutions of
a linear congruence (mod $20$)?
\begin{solution}
  By Theorem~1, the congruence $ax\equiv b\pmod{20}$ has $(20,a)$
  solutions, provided that $(20,a)\mid b$. So every divisor of $20$,
  along with $0$, is a possibility for the number of solutions: $0$,
  $1$, $2$, $4$, $5$, $10$, and $20$.
\end{solution}

\Problem6 Construct linear congruences modulo $20$ with no solutions,
just one solution, and more than one solution. Can you find one with
$20$ solutions?
\begin{solution}
  The linear congruence $2x\equiv3\pmod{20}$ has no solutions since
  $(2,20)\nmid3$. The congruence $3x\equiv1\pmod{20}$ has exactly one
  solution, $x = 7$. The congruence $2x\equiv8\pmod{20}$ has more than
  one solution: $x = 4$ and $x = 14$.

  The linear congruence $0x\equiv0\pmod{20}$ has $20$ solutions.
\end{solution}

\Problem7 Solve $9x\equiv4\pmod{1453}$.
\begin{solution}
  We have $9x\equiv4\equiv-1449\pmod{1453}$ and dividing by $9$ gives
  \begin{equation*}
    x\equiv-161\equiv1292\pmod{1453}.
  \end{equation*}
  This solution is unique, modulo $1453$.
\end{solution}

\Problem8 Solve $4x\equiv9\pmod{1453}$.
\begin{solution}
  Since $4x\equiv9\equiv-1444\pmod{1453}$, dividing by $4$ gives
  \begin{equation*}
    x\equiv-361\equiv1092\pmod{1453}.
  \end{equation*}
  This is the only solution.
\end{solution}
