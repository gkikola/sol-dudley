\chapter{Linear Congruences}

\section{Exercises}

\Exercise1 Construct congruences modulo $12$ with no solutions, just
one solution, and more than one solution.
\begin{solution}
  The congruence $3x\equiv4\pmod{12}$ has no solution since $3x$ is
  always congruent to $0$, $3$, $6$, or $9$ (mod $12$) and never
  $4$. The congruence $5x\equiv2\pmod{12}$ has one solution, $x =
  10$. The congruence $2x\equiv4\pmod{12}$ has two solutions, $x = 2$
  and $x = 8$.
\end{solution}

\Exercise2 Which congruences have no solutions?
\label{exercise:lin-cong:which-congruences-have-solutions}
\begin{enumerate}
\item $3x\equiv1\pmod{10}$,
\item $4x\equiv1\pmod{10}$,
\item $5x\equiv1\pmod{10}$,
\item $6x\equiv1\pmod{10}$,
\item $7x\equiv1\pmod{10}$.
\end{enumerate}
\begin{solution}
  Since $3\cdot7 = 21\equiv1\pmod{10}$, both of the congruences
  $3x\equiv1$ and $7x\equiv1\pmod{10}$ have a solution.

  The other three congruences do not have solutions.
\end{solution}

\Exercise3 After
Exercise~\ref{exercise:lin-cong:which-congruences-have-solutions}, can
you guess a criterion for telling when a congruence has no solutions?
\begin{solution}
  A necessary and sufficient condition that a congruence
  $ax\equiv b\pmod{m}$ has no solutions is that $(a,m)$ does not
  divide $b$. This will be proven in Lemma~1 in the text.
\end{solution}

\Exercise4 Solve
\begin{enumerate}
\item $8x\equiv1\pmod{15}$
  \begin{solution}
    Since $(8,15) = 1$, there is only one solution (by Lemma~2). We
    have $8x\equiv16\pmod{15}$ so that $x\equiv2\pmod{15}$.
  \end{solution}
\item $9x + 10y = 11$
  \begin{solution}
    From the equation we get the congruence
    $9x\equiv11\pmod{10}$. Since $11\equiv81\pmod{10}$, we have
    $9x\equiv81\pmod{10}$ from which we get $x\equiv9\pmod{10}$. This
    is the only solution to the congruence. Thus
    \begin{equation*}
      x = 9 + 10t
    \end{equation*}
    gives all possible values for $x$. Substituting this back into the
    equation, we get
    \begin{equation*}
      9(9 + 10t) + 10y = 11,
    \end{equation*}
    which gives
    \begin{equation*}
      y = -7 - 9t.
    \end{equation*}
    So the general solution is $x = 9 + 10t$ and $y = -7 - 9t$.
  \end{solution}
\end{enumerate}

\Exercise5
\label{exercise:lin-cong:sampler}
Determine the number of solutions of each of the following
congruences:
\begin{gather*}
  3x\equiv6\pmod{15},\qquad 4x\equiv8\pmod{15},\qquad 5x\equiv10\pmod{15}, \\
  6x\equiv11\pmod{15},\qquad 7x\equiv14\pmod{15}.
\end{gather*}
\begin{solution}
  We will use Lemma~3.

  $(3,15) = 3$ and $3\mid6$, so $3x\equiv6\pmod{15}$ has $3$
  solutions.

  $(4,15) = 1$, so $4x\equiv8\pmod{15}$ has only one solution.

  $(5,15) = 5$ and $5\mid{10}$, so $5x\equiv10\pmod{15}$ has $5$
  solutions.

  $(6,15) = 3$ but $3\nmid11$ so the congruence $6x\equiv11\pmod{15}$
  has no solutions.

  Finally, since $(7,15) = 1$, the congruence $7x\equiv14\pmod{15}$
  has one solution.
\end{solution}

\Exercise6 Find all of the solutions of $5x\equiv10\pmod{15}$.
\begin{solution}
  Since $(5,10) = 5$, we may divide by $5$ to get $x\equiv2\pmod3$. So
  the solutions modulo $15$ are $2$, $5$, $8$, $11$, and $14$.
\end{solution}

\Exercise7 Solve the rest of the congruences in
Exercise~\ref{exercise:lin-cong:sampler}.
\begin{solution}
  From $3x\equiv6\pmod{15}$ we get $x\equiv2\pmod{5}$, so that the
  three solutions modulo $15$ are $2$, $7$, and $12$.

  For $4x\equiv8\pmod{15}$ we get the unique solution $x = 2$.

  As we saw before, $6x\equiv11\pmod{15}$ has no solutions since
  $(6,15)\nmid11$.

  Lastly, for $7x\equiv14\pmod{15}$ we have the unique solution
  $x = 2$.
\end{solution}

\Exercise8 Verify that $52$ satisfies each of the three congruences,
$x\equiv1\pmod3$, $x\equiv2\pmod5$, and $x\equiv3\pmod7$.
\begin{solution}
  Since
  \begin{equation*}
    52 = 17\cdot3 + 1 = 10\cdot5 + 2 = 7\cdot7 + 3,
  \end{equation*}
  we see that each congruence is satisfied.
\end{solution}
