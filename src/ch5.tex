\chapter{Linear Congruences}

\section{Exercises}

\Exercise1 Construct congruences modulo $12$ with no solutions, just
one solution, and more than one solution.
\begin{solution}
  The congruence $3x\equiv4\pmod{12}$ has no solution since $3x$ is
  always congruent to $0$, $3$, $6$, or $9$ (mod $12$) and never
  $4$. The congruence $5x\equiv2\pmod{12}$ has one solution, $x =
  10$. The congruence $2x\equiv4\pmod{12}$ has two solutions, $x = 2$
  and $x = 8$.
\end{solution}

\Exercise2 Which congruences have no solutions?
\label{exercise:lin-cong:which-congruences-have-solutions}
\begin{enumerate}
\item $3x\equiv1\pmod{10}$,
\item $4x\equiv1\pmod{10}$,
\item $5x\equiv1\pmod{10}$,
\item $6x\equiv1\pmod{10}$,
\item $7x\equiv1\pmod{10}$.
\end{enumerate}
\begin{solution}
  Since $3\cdot7 = 21\equiv1\pmod{10}$, both of the congruences
  $3x\equiv1$ and $7x\equiv1\pmod{10}$ have a solution.

  The other three congruences do not have solutions.
\end{solution}

\Exercise3 After
Exercise~\ref{exercise:lin-cong:which-congruences-have-solutions}, can
you guess a criterion for telling when a congruence has no solutions?
\begin{solution}
  A necessary and sufficient condition that a congruence
  $ax\equiv b\pmod{m}$ has no solutions is that $(a,m)$ does not
  divide $b$. This will be proven in Lemma~1 in the text.
\end{solution}
